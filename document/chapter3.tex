%!TEX root = thesis.tex
\chapter{Active nematics on the surface of a torus}\label{c:3}
\section{Introduction}
Active materials are composed of mesogens that each can convert stored internal energy or ambient energy to kinetic energy, thus driving the material out-of equilibrium~\cite{RN237,RN238,RN40}.
Importantly, this is distinct from global driving mechanisms such as the application of a shear or the imposition of a field; in active materials, the material is driven out-of-equilibrium due to the intrinsic behavior of the constituent particles.
Thus, the materials cannot be understood within the framework of equilibrium statistical mechanics, as the individual particles are subject to forces independent of thermal fluctuations.
Since activity applies to each individual mesogen, the framework of ``active matter'' has been used to study self-organization and self-driven behavior in a variety of systems from flocks of starlings~\cite{RN239,RN240}, collections of robots~\cite{RN241}, colonies of fire ants~\cite{RN242}, cell growth and migration~\cite{RN51,RN160}, and self-propelled particles~\cite{RN168,RN38}.
Note that these examples of active matter span 6 orders of magnitude in size, with behavior such as flocking, giant number fluctuations, chaotic flows, and low-Reynolds number turbulence driven entirely by a single control parameter~\cite{RN237,RN238,RN40}. \\

Much like equilibrium nematics, active nematic materials are composed of individual mesogens that are anisotropic and thus can possess a nematic phase; however, the addition of activity to nematic order brings about an additional contribution to the interaction between the mesogens.
We note that with the exception of bacteria introduced in lyotropic liquid crystals~\cite{RN86}, to date, the majority of experimental and theoretical work on active nematics takes place in 2D.
Thus, to highlight the effect of activity on the nematic mesogens, we consider two active rod-like particles in 2D.
If the active nematic is ``extensile,'' the rods slide past each other; however, if the active nematic is ``contractile,'' the activity drives the rods to slide towards each other.
These interactions result in extensile active nematics being unstable to bend distortions while contractile active nematics are unstable to splay distortions~\cite{RN170,RN11}.
Thus, the homogeneously-aligned director state is unstable and the splay and bend instabilities often results in turbulent dynamics~\cite{RN7} and the continuous creation and annihilation of pairs of $s = \pm 1/2$ defects~\cite{RN11,RN8,RN3,RN27}.
These dynamics are driven by the $s = +1/2$ defects, where the activity coupled with the polar structure of the $s = +1/2$ defects causes the  $s = +1/2$ defects to act like self-propelled particles, with the propulsion direction depending on the type of activity~\cite{RN11,RN8}.
In contrast, due to the three-fold symmetry of $s = -1/2$ defects, $s = -1/2$ defects are not driven by activity and are only advected by interactions between the defects as well as by interactions with the surface.\\

Recent experimental studies with active nematics yielding self-regulated behaviors such as self-sustained oscillations~\cite{RN9}, spontaneous formation of morphological features such as kinks and protrusions~\cite{RN9,RN3}, and undirected motility~\cite{RN9,RN3} have acted as as stimulus to investigate how biological functionality emerges from the interplay between activity, the geometry of the system, and the structure of the internal phase~\cite{RN160,RN51,RN10}.
In this context, defects in active nematics could be harnessed to achieve life-like functionality, as defects are extremely sensitive to the intrinsic geometry of the space they inhabit.
This is easily seen in the connection between the total topological charge on a surface and the topology of the surface, as expressed in the Poincr\'e-hopf Theorem seen in Eq.~[INTRO].
However, the theory of defects in curved spaces extends this connection beyond topology, predicting that the free-energy of a collection of defects is sensitive to the local geometry through the Gaussian curvature~\cite{RN42}.
Importantly, this implies that 

In this chapter we consider a 2D extensile active nematic composed of microtubules driven by kinesin motors fueled by adenosine triphosphate (ATP) on the surface of a torus.
As a torus possesses regions of both positive and negative Gaussian curvature, this system gives us the ability to study not only the interplay between topological defects, curvature, and activity




\section{Making active nematic toroids}
\subsection{Active nematic formulation}
\subsection{Flat sample confirmation}
\subsection{Making toroidal droplets}

\section{Imaging active nematic toroids}
\subsection{Confocal setup and parameters}
\subsection{Passing confocal data into MATLAB}

\section{Determining director and defects}
\subsection{Coherence-enhanced diffusion filtering}
\subsection{Calculating the director}
\subsection{Finding defect location and topological charge}
\subsection{Edge charge}

\section{Measuring surface curvature}
\subsection{The Weingarten matrix}
\subsection{Iteratively-reweighted least squares}
\subsection{Fitting the Weingarten matrix on a surface}
\subsection{Correcting the surface normal vectors}
\subsection{Finding the area element}
\subsection{Validation on test surfaces}
\subsection{Measuring the curvature of toroidal droplets}

\section{Defect charge and curvature}
\subsection{Finding regions of specific integrated Gaussian curvature}
\subsection{Measuring defect charge in a specified region}
\subsection{Time-averaged defect charge as a function of integrated Gaussian curvature: defect unbinding}

\section{Defect number and curvature}
\subsection{Average defect density}
\subsection{Defect number distributions}

\section{Defect orientation and curvature}

\section{Comparison with numerical calculations}
\subsection{Simulation details}
\subsection{Matching simulation parameters}
\subsection{Estimates of material parameters}

\section{Conclusions}
