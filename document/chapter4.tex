%!TEX root = thesis.tex
\chapter{Homeotropic nematics confined in toroids and bent capillaries}

\section{Introduction}
A system with broken reflection symmetry, i.e. it cannot be superimposed onto its reflection using only translations and proper rotations, has a handedness and thus is chiral~\cite{RN175}.
Chirality can have important consequences for a system.
For example, as first shown by Pasteur, optical activity results from broken reflection symmetry, and the handedness of the system determines the direction of the rotation of the polarization of light\cite{RN291}.
In addition, chiral systems can exhibit structural color~\cite{RN308,RN307}, and can even be doped to form microlasers~\cite{RN305,RN306}.
Chiral systems can be formed from chiral building blocks, as in some photonic metamaterials~\cite{RN304}, or chirality can emerge via symmetry breaking in an achiral system\cite{RN294}.
This latter scenario is often studied in nematic liquid crystals where the mesogens are achiral\cite{RN297,RN296,RN298,RN295,RN299,RN193,RN46,RN192,RN191,RN293,RN302}.
Here, the symmetry-breaking is driven by elasticity; if the minimum energy state has a nontrivial twist distortion, then the system is chiral.

This simplest way to break the reflection symmetry is to proscribe it with boundary conditions, as in a twisted nematic cell.
In this scenario, the NLC is confined between two parallel plates with strong planar anchoring on each plate and the anchoring directions on each plate orthogonal to each other; consequently, the boundary conditions require $\mathbf{n}$ to twist by $\pi/2$ along a path between the plates.
Thus, the system has a persistent twist distortion, breaking reflection symmetry.
However, when the twist direction is not specifically set by the boundary, confined liquid crystals can exhibit spontaneous reflection symmetry breaking.
For example, consider a standard bipolar drop with degenerate planar anchoing, as shown schematically in Figure~\ref{f:2-OPMDrops}(A).
If the twist elastic constant $K_{22}$ and bend elastic constant $K_{33}$ are small enough compared to $K_{11}$, the splay elastic constant, the bipolar drop will twist, relieving some some of the splay with the less-costly twist and bend distortions~\cite{RN297,RN296,RN295}.
The criterion governing this instability is called the Williams criterion, $K_{11} > K_{22}+ 0.431 K_{33}$~\cite{RN297}.
Similarly, provided $K_{22}< K^c_{22}$, a NLC confined under homeotropic boundary conditions to a cylindrical capillary will exhibit a twisted escaped-radial (TER) configuration instead of the more common escaped-radial (ER) configuration~\cite{RN192}.

Recent work in our group with NLC confined to capillaries and toroids with degenerate planar boundary conditions have shown that the saddle-splay distortion can also enable a confined NLC to develop chirality~\cite{RN46,RN59,RN293}.
In this case, the saddle splay distortion drives the director at the interface to align along the smallest curvature of the interface according to Eq~\ref{e:2-K24SurfCouple}~\cite{RN59}; as a capillary can be thought of as a torus with an infinite aspect ratio, the criterion for a twisted ground state in both geometries is $K_{24} > K^c_{24}$~\cite{RN24,RN293}.
For a cylindrical capillary, a doubly-twisted state is the ground state if $K^c_{24}>K_{22}$.
However, by increasing the difference between the principle curvatures on the inside of the handle by decreasing $\xi$ and bending the cylinder into a torus, $K^c_{24}$ decreases and the amount of twist in a chiral ground state increases.
This is an example of how geometry can tune chirality.

Inspired by these results in planar-anchored nematic toroids, in this Chapter we consider a 5CB confined to a toroidal droplets under hometropic anchoring.
Despite the fact that the saddle-splay distortion does not enter into the energy minimization for homeotropic nematics, we find spontaneous reflection symmetry breaking and geometrically-controlled chirality.
As with homeotropically-anchored NLC capillaries when $K_{22} < K^c_{22}$~\cite{RN192}, we see that a TER configuration replaces the standard ER configuration.
However, we find TER configurations for $K_{22} > K^c_{22}$; we attribute this to the additional bend distortion introduced when taking a capillary with $\xi \rightarrow \infty$ to a torus with $\xi ~\mathcal(1--10)$.
This additional bend-distortion is relieved by a twist-distortion with a spontaneously-chosen handedness.
In addition, we see that the amount of twist varies inversely with $\xi$, similar to our previous findings in planar anchored toroids.




\section{Escaped radial and twisted escaped radial capillaries}
Consider NLC confined to a cylindrical capillary under homeotropic boundary conditions; in cylindrical coordinates $\{r,\theta,z\}$, we specify the boundary conditions with $\mathbf{n} = \hat{r}$ at the boundary.
Restricting ourselves to a radial director field, $\mathbf{n} = \hat{r}$ everywhere throughout the capillary will yield an $s = +1$ line defect running along $\hat{z}$ at the center of the capillary.
However, recall from Chapter~\ref{c:2} that integer-strength disclination lines are not topologically stable in a 3D NLC.
Similar to line defects in 2D nematics or wall defects in 3D nematics, the singular $s = +1$ distortion can be continuously deformed to form a nonsingular distortion [see Figure~\ref{f:2-Smearing}]~\cite{RN179,RN290,RN289}.
Here, the radial director field ``escapes into the third dimension'', such that  $\mathbf{n}_{z} \neq 0$, hence the name escaped-radial.
In the one-constant approximation, the ER director field is analytically solvable, with $\mathbf{n} = \{\sin(\Omega),0,\cos(\Omega)\}$, where $\Omega = 2 \arctan(r/R)$ gives the director angle measured off of $\hat{z}$, with $R$ the capillary radius~\cite{RN179,RN290}.
Provided $R \gtrsim 0.1$ $\upmu$m, the ER configuration is the ground state solution~\cite{RN194}; the costly splay distortion in the vicinity of the singular line is relieved by a loss-costly bend distortion, with the cost of the bend distortion varying inversely with $R$.

Now starting with an ER configuration in the 1-constant expression, let $K_{22}$ vary while keeping $K_{11} = K_{33} = K$.
When $K_{22} < K^c_{22} \approx 0.27K$~\cite{RN192}, the system can relieve some of the bend and spay distortions in the ER configuration by twisting, yielding the twisted escaped-radial configuration, where now $\mathbf{n}$ has components along $\hat{r}$, $\hat{\theta}$, and $\hat{z}$.
This configuration was first observed using Lyotropic Chromonic Liquid Crystals (LCLC), an aqueous dispersion of plank-like molecules that self-assemble into columnar aggregates; these aggregates then form a nematic at appropriate temperature and concentration~\cite{RN303}.
Since $K_{22} \sim 10^{-1}K$ for LCLC systems, they frequently exhibit spontaneous reflection symmetry breaking~\cite{RN192,RN191,RN293,RN193,RN302}.

The TER and ER configurations can be distinguished by OPM textures.
We fill cylindrical capillaries with 5CB and with an 31.5\% w/w aqueous solution of the LCLC Sunset Yellow (SSY).
To enforce homeotropic anchoring for the 5CB-filled capillaries, we first fill the capillary with $0.1\%$ w/w lecithin in hexane and then let the hexane dry, depositing the lecithin on the inner surface of the capillary.
For the SSY-filled capillaries, we coat the entire capillary with paralyene using chemical vapor deposition.


\subsection{Intensity profile and ratio}
By eye, it is easy to see that the ER and the TER textures are diffferent; this makes sense when we consider the director field in each configuration and its impact on the intensity profile across the capillary.
When the polarizer and analyzer (PA) are aligned along $\hat{z}$ and $\hat{r}$ projected onto the plane of the image, it is easy to see that an OPM texture of an ER capillary should have three regions of extinction corresponding to the edges and  to the center of the capillary.
In these regions, $\mathbf{n}$ is along the PA directions.
Between these three dark regions there are two bright regions, corresponding to locations where $\mathbf{n}$ has components along both $\hat{r}$ and $\hat{z}$.
This dark-bright-dark-bright-dark pattern is the signature of an ER capillary; now plotting the intensity profile across the capillary, we see that there are two clear intensity peaks corresponding the the bright regions, and an intensity minimum in the middle of the profile corresponding to the dark central region.

Conversely, when we look at the OPM texture for a TER configuration, we see that the central region is no longer dark, but now has an appreciable intensity comparable to the two ``bright'' regions flanking the central region.
In a TER configuration, the $\mathbf{n}_{\theta}$ component causes the projection of $\mathbf{n}$ onto the plane of the image to deviate from the PA directions, causing the transmitted intensity in the central region to grow.
Plotting the intensity profile across the capillary, we see that there is still a central minimum surrounded by two peaks; however, the central minimum in the TER configuration is much higher than the central minimum in an ER configuration.
We quantify this difference with an intensity ratio $I_{max}/I_{min}$, with $I_{max}$ the average intensity of the two peaks flanking the central minimum and $I_{min}$ is the intensity of the central minimum; our ER capillaries have $I_{max}/I_{min} \approx 4$ while our TER capillaries have $I_{max}/I_{min} \approx 1$.
We confirm that this phenomenon is not dependent on the LC chosen by comparing with the intensity profile of an ER SSY capillary; while the ground state in a homeotropic SSY capillary is a TER configuration, just after being filled the SSY capillary will have a transient ER configuration that spontaneously breaks reflections symmetry after a few minutes.
As with the ER textures from the 5CB capillaries, $I_{max}/I_{min} \approx 4$ in the ER SSY capillary.




\section{Nematic liquid crystals in toroids}
We make stable toroidal droplets as detailed in Chapter~\ref{c:3}, with 5CB as our inner liquid and an aqueous yield-stress material.
Our yield-stress material for homeotropic nematic toroids is similar to that used in our prior work with planar-anchored nematic toroids~\cite{RN46}, except here, we have replaced the Polyvinyl alcohol used to enforce degenerate planar anchoring with Sodium Dodecyl Sulfate (SDS) to enforce homeotropic anchoring.
The yield-stress material consists of (i) 1.5\% w/w polyacrylamide microgels (Carbopol ETD 2020), (ii) 30\% w/w Ethanol, (iii) 3\% w/w Glycerol, (iv) 25\% w/w 32 mM SDS in ultrapure water, and (v) 40.5\% w/w ultrapure water.
Thus, the final mixture has 8 mM SDS, a concentration that yields strong homeotropic anchoring.
To make the yield-stress material, we mix all the ingredients together, leave the dispersion for $\sim 24$ hrs to allow the Carbopol to hydrate, and then neutralize the mixture using 2M NaOH until the pH$\approx 7$.
Neutralizing the dispersion causes the polyacrylamide microgels to swell, increasing both the yield-stress and the transparency~\cite{RN46,RN47}.


\subsection{Measuring the intensity profile and aspect ratio}
We view our toroidal droplets from the top under both bright-field and with crossed polarizers.
In the toroidal coordinate system $\{r,\theta,\varphi \}$, the azimuthal angle $\varphi$ corresponds to a polar angle in the plane of the image.
Thus, with PA aligned along the vertical and horizontal in the image, $\varphi = 0,\pi/2,\pi,3\pi/2$ gives the locations in the torus where $\hat{phi}$ is aligned along PA.
In these regions, the intensity pattern resembles that of an escaped radial texture with $\hat{z}$ aligned along PA, with the characterisitc dark-bright-dark-bright-dark pattern.
To obtain the both intensity profile across the tube where $\hat{\varphi}$ is aligned along PA and the aspect ratio of the torus, we turn to a custom MATLAB script.

From a top-view image, we start by selecting points along the inner and outer contours of the toroid in the image.
We then fit the points on each contour to a circle, giving us the radii $R_{in}$, $R_{out}$ and centers $\mathbf{p}_{in}$, $\mathbf{p}_out$ of the inner and outer contours, respectively.
For a given torus, we can then calculate the aspect ratio $\xi = R_0/a = (R_{out}+R_{in})/(R_{out}-R_{in})$, with the central ring radius $R_0=(R_{out}+R_{in})/2$ and the tube radius $a = (R_{out}-R_{in})/2$.
From the center of the central ring $\mathbf{p}_0 = (\mathbf{p}_{in}+p_{out})/2$, we calculate $\varphi$ everywhere in the image, enabling us to measure the intensity profile across the tube for a given $\varphi$.
We parameterize the intensity profile with $\rho$, where $\rho = 0$ at the inner contour and $\rho = 1$ at the outer contour.


\subsection{Large aspect ratio toroids}
\subsection{Small aspect ratio toroids}

\section{Simulating polarized optical microscopy textures for twisted escaped radial director configurations}
\subsection{Jones Calculus}
\subsection{Validation using spherical droplets}
\subsection{Planar-anchored nematic toroids}
\subsection{Comparison with homeoetropic-anchored nematic toroids}
\subsection{Intensity ratio as a function of twist parameter}

\section{Nematic liquid crystals in bent capillaries}
\subsection{Making bent capillaries}
\subsection{Measuring planar curvature}
\subsection{Measuring the intensity profile}
\subsection{Comparison with toroids}

\section{Conclusions}
