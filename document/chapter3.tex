%!TEX root = thesis.tex
\chapter{Active nematics on the surface of a torus}\label{c:3}
\section{Introduction}
Active materials are composed of mesogens that each can convert stored internal energy or ambient energy to kinetic energy, thus driving the material out-of equilibrium~\cite{RN237,RN238,RN40}.
Importantly, this is distinct from global driving mechanisms such as the application of a shear or the imposition of a field; in active materials, the material is driven out-of-equilibrium due to the intrinsic behavior of the constituent particles.
Thus, the materials cannot be understood within the framework of equilibrium statistical mechanics, as the individual particles are subject to forces independent of thermal fluctuations.
Since activity applies to each individual mesogen, the framework of ``active matter'' has been used to study self-organization and self-driven behavior in a variety of systems from flocks of starlings~\cite{RN239,RN240}, collections of robots~\cite{RN241}, colonies of fire ants~\cite{RN242}, cell growth and migration~\cite{RN51,RN160}, and self-propelled particles~\cite{RN168,RN38}.
Note that these examples of active matter span 6 orders of magnitude in size, with behavior such as flocking, giant number fluctuations, chaotic flows, and low-Reynolds number turbulence driven entirely by a single control parameter~\cite{RN237,RN238,RN40}. \\

Much like equilibrium nematics, active nematic materials are composed of individual mesogens that are anisotropic and thus can possess a nematic phase; however, the addition of activity to nematic order brings about an additional contribution to the interaction between the mesogens.
We note that with the exception of bacteria introduced in lyotropic liquid crystals~\cite{RN86}, to date, the majority of experimental and theoretical work on active nematics takes place in 2D.
Thus, to highlight the effect of activity on the nematic mesogens, we consider two active rod-like particles in 2D.
If the active nematic is ``extensile,'' the rods slide past each other; however, if the active nematic is ``contractile,'' the activity drives the rods to slide towards each other.
These interactions result in extensile active nematics being unstable to bend distortions while contractile active nematics are unstable to splay distortions~\cite{RN171,RN170,RN11}.
Thus, the splay and bend instabilities make the homogeneously-aligned director state unstable such that the steady-state of an active nematic is often turbulent~\cite{RN7} and full of pairs of $s = \pm 1/2$ defects that are continuously created and annihilated~\cite{RN11,RN8,RN3,RN27,RN135,RN86}.
These turbulent dynamics are driven by the $s = +1/2$ defects, where the activity coupled with the polar structure of the $s = +1/2$ defects causes the  $s = +1/2$ defects to act like self-propelled particles, with the propulsion direction depending on the type of activity~\cite{RN11,RN8}.
In contrast, due to the three-fold symmetry of $s = -1/2$ defects, $s = -1/2$ defects are not driven by activity and are only advected by interactions between the defects as well as by interactions with the surface.\\

Recent experimental studies with active nematics yielding self-regulated behaviors such as self-sustained oscillations~\cite{RN9}, spontaneous formation of morphological features such as kinks and protrusions~\cite{RN9,RN3}, and undirected motility~\cite{RN9,RN3} have acted as as stimulus to investigate how biological functionality emerges from the interplay between activity, the geometry of the system, and the structure of the internal phase~\cite{RN160,RN51,RN10}.
In this context, defects in active nematics could be harnessed to achieve life-like functionality, as defects are extremely sensitive to the intrinsic geometry of the space they inhabit.
This is easily seen in the connection between the total topological charge on a surface and the topology of the surface, as expressed in the Poincr\'e-hopf Theorem seen in Eq.~[INTRO].
However, the theory of defects in curved spaces extends this connection beyond topology, predicting that the free-energy of a collection of defects is sensitive to the local geometry through the Gaussian curvature~\cite{RN42}.
Importantly, this implies that even though there have been new discoveries as a result of examining defect structures on spheres~\cite{RN45,RN106,RN26,RN110,RN105,RN76,RN101,RN165}, the Gaussian curvature of a sphere is constant such that the impact of curvature can only enter through the sphere radius.
This is is born out in the size-dependent onset of grain-boundary scars in colloidal crystals on the surface of emulsion drops~\cite{RN26,RN110} and the degeneracy in the orientation of the 4 $s = +1/2$ defects in nematic shells~\cite{RN45}.
Even when considering an active nematic on a sphere and the 4 $s = +1/2$ defects become mobile, the defect dynamics can be explained without any mention of Gaussian curvature~\cite{RN9}. \\

Despite theoretical interest in the interplay between varying Gaussian curvature and topological defects, with few exceptions~\cite{RN84,RN25,RN73,RN81}, there has been little experimental work with defects on surfaces where the Gaussian curvature is non-constant.
More notably, there has been no work where the defects have the option to explore regions with both positive and negative Gaussian curvature.
In this scenario with nematic order, the topological charge is expected to unbind such that $s = +1/2$ defects are attracted to regions of positive Gaussian curvature and $s=  -1/2$ defects are attracted to regions of negative Gaussian curvature~\cite{RN17,RN19,RN22}.
In this chapter we consider a 2D extensile active nematic composed of microtubules driven by kinesin motors fueled by adenosine triphosphate (ATP) on the surface of a torus.
Since the torus the torus has a handle, it has a genus of one and thus according to the Poincar\'e-Hopf Theorem written in Eq.~[INTRO] must have vanishing net topological charge.
We note that a nematic on a torus can satisfy this condition by either being defect-free or by having the same amount of positive and negative charge.
While strong spatial confinement or low activity can suppress the spontaneous formation of defect pairs in an active nematic~\cite{RN9}\fxnote{cite green}, we perform our experiments in the turbulent regime such that the torus is always populated with a sea of constantly moving $s = \pm 1/2$ defects that are dynamically created and annihilated.\\

Despite such chaotic and highly nonequilibrium dynamics, we find that on average, topological defects unbind and segregate in regions of oppositely-signed Gaussian curvature.
Notably, due to the chaotic dynamics and the averaging process, the topological charge ceases to be a discrete variable and instead approaches a continuous distribution.
In addition, contrary to equilibrium predictions~\cite{RN36,RN19,RN22,RN20,RN78}, we find that this active unbinding depends only on the local geometry and is independent of the system size and aspect ratio.
When also consider the defect species themselves and find that the average defect density depends inversely on Gaussian curvature over the majority of the toroidal surface, deviating only near the inside and outside of the handle.
Interestingly, near the inside and outside of the handle, we see that the defects themselves, on average, begin to exhibit orientational order.
While $s = +1/2$ defects have been shown to order themselves into a nematic phase in flat space, we find that the $s = +1/2$ defects on a torus exhibit polar order instead~\cite{RN27,RN6}.
We perform our studies with two different ATP concentrations and thus 2 different activities and find that the defect unbinding, the defect density, and the orientational order of the defects all depend on activity.
A numerical integration of the equation of motion of active nematic defects~\cite{RN11,RN8,RN9} performed by Luca Giomi and Dan Pearce at the University of Leiden confirms our experimental results, and further illustrating that the defect unbinding can even be suppressed in the limit of high activity.
Furthermore, by using topological defects as micro-rheological tracers and quantitatively comparing our experimental and theoretical results, we are able to estimate the Frank elastic constant, the active stress, and the defect mobility of a microtubule-kinesin active nematic liquid crystal.
Overall, out results not only confirm the theory of topological defects on curved surfaces, but also demonstrate the surprising phenomenology that arises from adding activity to the interplay between geometry, topology, and order. Out work thus provides insights into the physics of partially ordered active matter and introduces a new avenue for the quantitative mechanical characterization of active fluids.\\


\section{Making active nematic toroids}
\subsection{Active nematic formulation}
We use the mictotubule-kinesin active nematic system pioneered by the Dogic Group at Brandeis.
\subsection{Flat sample confirmation}
\subsection{Making toroidal droplets}

\section{Imaging active nematic toroids}
\subsection{Confocal setup and parameters}
\subsection{Passing confocal data into MATLAB}

\section{Determining director and defects}
\subsection{Coherence-enhanced diffusion filtering}
\subsection{Calculating the director}
\subsection{Finding defect location and topological charge}
\subsection{Edge charge}

\section{Measuring surface curvature}
\subsection{The Weingarten matrix}
\subsection{Iteratively-reweighted least squares}
\subsection{Fitting the Weingarten matrix on a surface}
\subsection{Correcting the surface normal vectors}
\subsection{Finding the area element}
\subsection{Validation on test surfaces}
\subsection{Measuring the curvature of toroidal droplets}

\section{Defect charge and curvature}
\subsection{Finding regions of specific integrated Gaussian curvature}
\subsection{Measuring defect charge in a specified region}
\subsection{Time-averaged defect charge as a function of integrated Gaussian curvature: defect unbinding}

\section{Defect number and curvature}
\subsection{Average defect density}
\subsection{Defect number distributions}

\section{Defect orientation and curvature}

\section{Comparison with numerical calculations}
\subsection{Simulation details}
\subsection{Matching simulation parameters}
\subsection{Estimates of material parameters}

\section{Conclusions}
