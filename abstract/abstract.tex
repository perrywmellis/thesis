\documentclass[11pt]{article}   %12 point font for Times New Roman
\usepackage{setspace}  %for images and plots
\usepackage[letterpaper, left=1.25in, right=1.25in, top=1.25in, bottom=1.25in]{geometry}

\begin{document}
\onehalfspacing
\begin{centering}
Nematic Materials in Curved Spaces \\[11 pt]

Perry W. Ellis \\[11 pt]

XX Pages \\[11 pt]

Directed by Dr. Alberto Fern\'andez-Nieves \\[22 pt]
\end{centering}
When confined by curved surfaces or volumes, ordered materials experience geometric frustration.
This frustration causes unavoidable distortions in the material and can often be relieved by nucleating one or more defects in the order.
In this thesis, we focus on nematic materials and investigate the role of geometry in the interplay between order and confinement.
We first consider a 2D nematic confined to the surface of a toroidal droplet; the nematic is active, meaning that the individual particles have their own source of internal energy.
The activity then drives the material out of equilibrium at the individual particle level, resulting in a nematic that is filled with opposite-signed topological defects that are constantly moving and undergoing creation and annihilation.
Despite the fact that we have a nonequilibrium material, we find that the defects couple to the underlying curvature; on average, the positive defects migrate towards the outside of the hole of the torus and the negative defects migrate towards the inside of the hole.
This curvature-induced defect unbinding was originally predicted for equilibrium crystals and nematics on a torus.
When we compare our results to the equilibrium predictions as well as to computer simulations of our active nematic, we find that adding activity to order qualitatively resembles bringing an equilibrium system to the high temperature limit.
However, there are significant differences with equilibrium nematics, which we highlight.
We next consider a nematic liquid crystal (NLC) confined to toroidal droplets with homeotropic anchoring.
We find a twisted ground state, where the amount of twist depends on the ratio of the torus ring radius to its tube radius.
Experiments with a NLC confined to straight and bent cylindrical capillaries under homeotropic boundary conditions reveal that the twist is a response to the additional curvature induced when deforming a cylinder of homeotropic nematic into a torus.
Lastly, we consider a NLC confined to a capillary bridge under homeotropic anchoring such that the topology requires the presence of a defect in the bulk.
We perform experiments with waist-shaped and barrel-shapes bridges and find that waist-shaped bridges contain hyperbolic defects while barrel-shaped bridges contain radial defects.
In addition, we find that the ratio of the bridge height to its width determines whether the defect is a ring defect or a point defect.
\end{document}
