%!TEX root = thesis.tex
\chapter{Homeotropic nematics confined in capillary bridges}

\section{Introduction}
The research presented so far has focused on nematic materials confined by toroidal surfaces.
Here we consider a NLC confined to a topologically spherical volume with homeotropic boundary conditions.
As established in Chapter~\ref{c:1}, the volume must contain a total hedgehog charge $|d| = 1$.
There are a myriad of possible defect configurations for such a topologically-confined system, from the simple configuration of a single radial hedgehog~\cite{RN177}, to far more complicated defect structures involving chains of hedgehog defects~\cite{RN150} or even knotted and entangled line-defect structures~\cite{RN277,RN278}.
All of these well-understood~\cite{RN93,RN275,RN276} examples involve geometrically spherical boundaries, where the confinement can only be modified through changing the sphere radius.
However, the role of shape when confining NLC in geometries with more than one characteristic lengthscale is not completely understood.

Consider the case of a cylindrical geometry of aspect ratio $\Gamma = 2 R/H$, where $R$ is the radius of the cylinder and $H$ is its height.
With this notation, the classic case of a cylindrical capillary corresponds to $\Gamma \ll 1$. The $\Gamma \gg 1$ situation corresponds to confinement between narrowly separated plates.
When $\Gamma \sim \mathcal{O}\left( 1 \right)$, the equilibrium defect configuration undergoes a transition from a ring defect, found when $\Gamma \gg 1$, to the point defect, seen when $\Gamma \ll 1$.
Prior experimental work investigating this transition used liquid crystal capillary bridges made with 5CB, and reported a transition between a claimed hyperbolic ring defect [see Figs.~1\emph{A},\emph{C}] and a claimed radial point defect [see Figs.~1\emph{E},\emph{F}]~\cite{RN139,RN147}.
% NLC pentylcyanobiphenyl (5CB)
However, we note that Refs.~\cite{RN139,RN147} only observed the bridge structures from above, where the radial and hyperbolic defect structures look the same, as demonstrated schematically in Fig. 1\emph{C} and Fig. 1\emph{F} for ring defects and point defects, respectively.
Prior theoretical work used computational modeling to explore the defect configuration within a cylindrical bridge as a function of $\Gamma$ and $K_{11}/K_{33}$~\cite{RN138,RN144}.
% where $K_{11}$ and $K_{33}$ are the Frank elastic constants corresponding to splay and bend distortions, respectively
For 5CB, which has $K_{11}/K_{33} = 0.74$, they predicted, in contrast to the claims in Refs.~\cite{RN139,RN147}, that the bridge should transition between a radial ring defect and a hyperbolic point defect.

In this Chapter, we address the conflict described in the previous paragraph and perform both experiments and computations pertaining to a confined NLC within a capillary bridge sandwiched between two parallel plates of adjustable separation and hence of varying $\Gamma$.
By observing our experimental bridges from both the top and the side, and comparing our observations with results from our computations, we find that that shape of the free surface controls whether the defect is radial or hyperbolic: waist-like bridges contain hyperbolic defects, and barrel-like bridges contain radial defects.
In addition, we find good agreement between experiment and theory for the critical aspect ratio $\Gamma_c$ at which the defect in the bridge undergoes a transition between a ring defect and a point defect.
Finally, we see that this transition is hysteretic, due to the metastability of the point defect.
Our results clarify the prior discrepancies, and at the same illustrate how shape and elasticity dictate defect structure in confined homeotropic nematics.




\section{Making capillary bridges}
To make a capillary bridge, we confine 5CB between two parallel glass microscope slides.
Prior to use, the slides were dip-coated with $0.1\%$ w/w lecithin (granular, Acros) in hexane ($98.5\%$ purity, BDH) and left to dry to enforce homeotropic anchoring~\cite{RN140}.
We set up an experiment to view a bridge from the top by first placing both microscope slides stacked on top of each other on the microscope stage.
We then epoxy the top plate to a rod affixed to a micromanipulator such that we can adjust the distance between the slides.
Note that this simple protocol ensures that the two microscope slides are parallel to each other and to the microscope stage.
After the epoxy hardens, we raise the top slide and use a glass capillary to place a $\sim$nl-volume drop of 5CB onto the bottom plate.
We then bring the top plate down until it makes contact with the sessile droplet and forms a capillary bridge.

To set up an experiment to view a bridge from the side, we place an uncoated glass slide on the microscope stage to act as a base, and then place a lecithin-coated glass slide vertically on the base and use a pair of blocks to hold it in place.
We then epoxy the lecithin-coated slide to the base, applying epoxy to only one side of the joint between the base and the lecithin-coated slide.
Once the epoxy has hardened, we remove the blocks and place the second lecithin-coated glass slide vertically on the base and flush against the previously-epoxied glass slide.
We then use the rod attached to the micromanipulator to hold the two vertical slides flush while we epoxy the second lecithin-coated glass slide to the rod.
This protocol ensures that the two lecithin-coated glass slides are parallel to each other and perpendicular to the base.
After the epoxy hardens, we use the mircomanipulator to move the adjustable plate as far as possible from the fixed plate and place a $\sim$nl-volume drop of 5CB onto the fixed vertical plate as close to the base as we can.
Finally, we bring the adjustable plate closer to the fixed plate until it makes contact with the sessile drop and forms a capillary bridge.
The final experimental setup for both a top view and a side view is depicted schematically from the side in Figure~\fxnote{F1}(A,B) and in example images in Figure~\fxnote{F1}(C,D), respectively.

As described, this procedure will yield a capillary bridge where the free-surface is in contact with air.
To make a capillary bridge with the free surface in contact with water, we first make a bridge as described above and then pipette a drop of water near the edge of the parallel plates and let capillary action fill the gap between the plates.
The water contains 8 mM SDS to enforce homeotropic anchoring.




\section{Shape of capillary bridges}
\subsection{Measuring the shape}
We begin by viewing the bridges from the side and characterizing their shapes.
Bridges with air as an outer medium have a waist-like shape with negative Gaussian curvature everywhere on the free surface and bridges with water and SDS as the outer medium have a barrel-like shape.
For some example bridges, we record the shape contours on both the left and the right of the bridge as a function of $\Gamma$.
We then plot the contours normalized by the bridge height, with the left contours reflected about the vertical axis, and all contours shifted so that their lowest point corresponds to the origin.
For both the barrels and the waists, the contours all approximately have the same shape regardless of aspect ratio or experiment.


\subsection{Constant mean curvature surfaces}
To address the origin of the shape, we consider the relevant forces: the gravitational force $|\mathbf{F}_g| \sim \rho g R^2 H$; the surface tension force $|\mathbf{F}_{\gamma}| \sim \gamma H$; and the nematic elasticity force $|\mathbf{F}_K| \sim K$.
The surface tension, density, and Frank elastic constant of 5CB are equal, respectively, to $\gamma \approx 30$ mN/m, $\rho \approx 1$ g/mL, and $K \approx 10^{-11}$ N.
We compare these forces via two dimensionless groups: the Bond number $\rm{Bo} = \dfrac{|\mathbf{F}_g|}{|\mathbf{F}_{\gamma}|} = \dfrac{\rho g R^2}{\gamma} \sim  \mathcal{O}\left (10^{-2} \right)$, and the elasticity group,
 $\dfrac{|\mathbf{F}_{\gamma}|}{|\mathbf{F}_K|} =  \dfrac{\gamma H}{K} \sim \mathcal{O}\left (10^{5} \right )$, where we have taken $H = R = 100$ $\mu$m as representative values.
As $|\mathbf{F}_{\gamma}|$ is the dominant force, the mean curvature of the free surface of the bridge must be constant~\cite{RN178}.
Thus, the contours for a surface of revolution in cylindrical coordinates $\{r,\phi, z\}$ should conform to the expression:
\begin{equation}
  \Delta P = 2 \gamma \left ( \frac{\textrm{d}\Theta}{\textrm{d}s} + \frac{\sin \theta}{r} \right ) = \textrm{cons't},\label{e:5-ConsMeanCurv}
\end{equation}
where $\Delta P$ is the Laplace Pressure from Eq.~\ref{e:3-LapPres}, $\gamma$ is the surface tension, and $\Theta$ and $s$ are the elevation angle and arclength parameter, respectively, as defined in Figure~\fxnote{Fig here}.
This further implies that the contact angle $\theta_0$ between the lecithin-coated glass slide, the outer medium, and the 5CB should set the shape~\cite{RN178}.

To confirm this, we consider each example bridge and calculate constant-mean curvature contours that capture the envelope of the observed contours for each bridge.
Each calculated contour is completely determined by a contact angle and a constant mean curvature value.
We average the contact angles for all the contours from the waists and barrels to get blah and blah


\subsection{Contact angle comparisons}
We compare the contact angles determined from our calculated contours with contact angles measured from sessile droplets with both air and water as the outer medium.




\section{Defect structure transitions}
\subsection{Defect transitions in a waist}
\subsection{Defect transitions in a barrel}

\section{Measuring defect conformation using fluorescence microscopy}
\subsection{Theoretical overview of fluorescence}
\subsection{Experimental realization}
\subsection{Validation in planar nematic cells}
\subsection{Validation in spherical droplets and capillaries}
\subsection{Defect conformation in a waist}
\subsection{Defect conformation in a barrel}

\section{Comparison with numerical calculations}

\section{Conclusions}
