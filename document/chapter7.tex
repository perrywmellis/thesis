%!TEX root = thesis.tex
\chapter{Summary, conclusions, and future work}

\section{Summary and conclusions}
Confining ordered materials can lead to interesting phenomonology due to the interplay between the order and the confinement geometry.
For the nematic materials used in this Thesis, it is often the curvature of the confining volume or surface that affects the nematic order.
This is a consequence of the fact that nematic materials possess only orientational order; with the order defined by the director $\mathbf{n}$, the free energy has order $(\nabla \mathbf{n})^2$, indicating that curvature in the director field costs energy.
The influence of curvature on a nematic material is even more explicit when considering a 2D nematic constrained to lie on a surface.
In this situation, the free energy can be written to resemble that of a multicomponent plasma, with defects in the nematic acting like discrete charges in the plasma and the Gaussian curvature of the surface as background charge in the plasma  [see Eq~\ref{e:1-TopTheoryofDefects}].
Thus, if the surface has both positive and negative Guassian curvature and the nematic has both positive and negative defects, the defects are predicted to segregate, with the positive defects migrating to the regions of positive Gaussian curvature and vice versa.

We explore this situation experimentally using an active polymeric nematic depleted to the surface of a toroidal droplet.
Due to the activity, the nematic is filled with pairs of constantly creating and annihilating $s = \pm 1/2$ defects, with the $s = +1/2$ defects acting as self-propelled particles driving the nematic into a turbulent state.
We measure the time-averaged topological charge in regions on the toroidal droplet and find that the average charge varies linearly with the integrated Gaussian curvature in the region.
The slope of this relationship is positive, indicating that our system exhibits defect unbinding.
In contrast to predictions for a system at equilibrium, we find that the active unbinding depends only on the local geometry and is insensitive to the size and aspect ratio of our toroidal droplets.
Comparing our experimental results to a numerical integration of the equations of motion of active nematic defects further illustrates that the defect unbinding also depends on the defect number density and that the unbinding can even be suppressed in the limit of high activity.
Finally, by using topological defects as micro-rheological tracers and quantitatively comparing our experimental and theoretical results, we are able to estimate the Frank elastic constant, the active stress, and the defect mobility of a microtubule-kinesin active NLC.

Overall, our results not only confirm the theory of topological defects on curved surfaces, but also demonstrate how adding activity to an ordered material changes and enriches equilibrium expectations.
For example, because the active unbinding is driven solely by local interactions, we see that a combination of activity and curvature can be used to guide defects in ordered materials.
In addition, our work introduces a new avenue for the quantitative mechanical characterization of active fluids.

So far we have only examined the behavior of the average topological charge $\overbar{s}_{\Theta} = (\overbar{N}^{+}_{\Theta} - \overbar{N}^{-}_{\Theta})/2$, with $\overbar{N}^{\pm}_{\Theta}$ the time-averaged number of $s = \pm1/2$ defects in a region $\Theta$ on the active nematic toroid.
This is only one of the things that our setup can explore.
For example, we have preliminary data showing the time-averaged number density $\overbar{N}_{\Theta}/A_{\Theta} = (\overbar{N}^{+}_{\Theta} + \overbar{N}^{-}_{\Theta})/A_{\Theta}$ depends on not only the local curvature, but also the aspect ratio of the toroid.
It is not clear why the topological charge depends only on local interactions while the defect number density depends on the size and shape of the toroidal droplet.
In addition, there is work with this active nematic in flat space showing that the $s = +1/2$ defects can themselves assemble to form a nematic phase, where $S$ associated to this higher-order nematic phase grows as $\overbar{N}_{\Theta}$ grows.
However, our preliminary data suggest that on a toroid, the $s = +1/2$ defects do not form a nematic phase but instead a polar phase, and that the strength of this polar phase grows as the $\overbar{N}_{\Theta}$ decreases.
These are just two examples of future directions exploring to the role of curvature and activity in partially ordered matter.

We also consider a NLC confined to toroidal droplets and bent capillaries under homeotropic boundary conditions.
We observe spontaneous reflection symmetry breaking due to a twist distortion relieving the energetic cost of two competing bend distortions.
The competition between the distortions is given by the local aspect ratio, $\xi$, comparing the radii of curvature of the two bend distortions.
Thus, $\xi$ also governs the amount of twist in the system, resulting in a geometrically-tuned chirality.
This relationship between twist and $\xi$ is similar to previous results in our group with NLC confined to toroidal droplets with degenerate planar anchoring, showing that tuning a ratio of curvatures to control chirality in NLC does not depend on the anchororing.

Lastly, we explore the equilibrium defect structure in NLC confined to capillary bridges under homeotropic boundary conditions.
We find that the defect structure in our bridges depends on both the shape of the bounding surface as well as the aspect ratio of the bridge.
The aspect ratio determines whether the bridge contains a ring defect or a point defect, and the boundary shape determines whether the defect is radial or hyperbolic, with waist-like shapes containing hyperbolic defects and barrel-like shapes containing radial defects.
In addition, we find that in a waist structure the point defect can be metastable, causing the transition between a ring defect and a point defect to exhibit hysteresis.
We compare with numerical calculations and find good agreement with our experiments.
Our work the shows that shape can be used to influence and control the equilibrium defect states in confined NLC under homeotropic boundary conditions.

Interestingly, the numerical calculations predict that a cylinder-like structure should contain hyperbolic rings but radial points.
The specific pathway for this transition is unclear and would be an interesting direction for future work.
Further interesting results would also be expected if the shape of the bridge is not fixed by surface tension, but can instead change and contribute to the free energy minimization \cite{RN12}.
Our work is thus one of many interesting studies that can be performed with nematic bridges to probe how shape and elasticity dictate the equilibrium defect structure of the liquid crystal.

\section{Defect orientation on curved surfaces: current status}
\subsection{Theory}
\subsection{Experiment}
\subsection{Future Work}

\section{Determining the saddle-splay elastic constant}
\subsection{Prior estimates in tori}
\subsection{Results from spherical drops}
\subsection{Distinguishing $K_{24}$ and surface contributions using twist angle and temperature}
