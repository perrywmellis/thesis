%!TEX root = thesis.tex
\chapter{Homeotropic nematics confined in toroids and bent capillaries}

\section{Introduction}
A system with broken reflection symmetry, i.e. it cannot be superimposed onto its reflection using only translations and proper rotations, has a handedness and thus is chiral.
Chirality can have important consequences for a system.
For example, as first shown by Pasteur, optical activity results from broken reflection symmetry, and the handedness of the system determines the direction of the rotation of the polarization of light.
In addition, chiral systems can exhibit structural color, and can even be doped to form microlasers.
Chiral systems can be formed from chiral building blocks, as in some photonic metamaterials, or chirality can emerge via symmetry breaking in an achiral system.
This latter scenario is often studied in nematic liquid crystals where the mesogens are achiral.
Here, the symmetry-breaking is driven by elasticity; if the minimum energy state has a nontrivial twist distortion, then the system is chiral.

This simplest way to break the reflection symmetry is to proscribe it with boundary conditions, as in a twisted nematic cell.
In this scenario, the NLC is confined between two parallel plates with strong planar anchoring on each plate and the anchoring directions on each plate orthogonal to each other; consequently, the boundary conditions require $\mathbf{n}$ to twist by $\pi/2$ along a path between the plates.
Thus, the system has a persistent twist distortion, breaking reflection symmetry.
However, when the twist direction is not specifically set by the boundary, confined liquid crystals can exhibit spontaneous reflection symmetry breaking.
For example, consider a standard bipolar drop with degenerate planar anchoing, as shown schematically in Figure~\ref{f:2-OPMDrops}(A).
If the twist elastic constant $K_{22}$ and bend elastic constant $K_{33}$ are small enough compared to $K_{11}$, the splay elastic constant, the bipolar drop will twist, relieving some some of the splay with the less-costly twist and bend distortions.
The criterion governing this instability is called the Williams criterion, $K_{11} > K_{22}+ 0.431 K_{33}$.
Similarly, provided $K_{22}< 0.27 K_{11} = 0.27K_{33}$, a NLC confined under homeotropic boundary conditions to a cylindrical capillary will exhibit a twisted escaped-radial (TER) configuration instead of the more common escaped-radial (ER) configuration.

Recent work with NLC confined to capillaries and toroids with degenerate planar boundary conditions have shown that the saddle-splay distortion can also enable a confined NLC to develop chirality.
In this case, the saddle splay distortion drives the director at the interface to align along the smallest curvature of the interface according to Eq~\ref{e:2-K24SurfCouple}; as a capillary can be thought of as a torus with an infinite aspect ratio, the criterion for a twisted ground state in both geometries is $K_{24} > K^c_{24}$.
For a cylindrical capillary, a doubly-twisted state is the ground state if $K^c_{24}>K_{22}$.
However, by increasing the difference between the principle curvatures on the inside of the handle by decreasing $\xi$ and bending the cylinder into a torus, $K^c_{24}$ decreases and the amount of twist in a chiral ground state increases.
This is an example of how geometry can tune chirality.

Inspired by these results in planar-anchored nematic toroids, in this Chapter we consider a NLC confined to a toroidal droplets under hometropic anchoring.
Despite the fact that the saddle-splay distortion does not enter into the energy minimization for homeotropic nematics, we find spontaneous reflection symmetry breaking and geometrically-controlled chirality.
As with homeotropically-anchored NLC capillaries when $K_{22} < 0.27 K_{11} = 0.27K_{33}$, we see that a TER configuration replaces the standard ER configuration.
However, we find TER configurations for $K_{22} > 0.27 K_{11} = K_{33}$; we attribute this to the additional bend distortion introduced when taking a capillary with $\xi \rightarrow \infty$ to a torus with $\xi ~\mathcal(1--10)$.
The additional bend is relieved by a twist-distortion with a spontaneously-chosen handedness.
In addition, we see that the amount of twist varies inversely with $\xi$, similar to our previous findings in planar anchored toroids.




\section{Escaped radial and twisted escaped radial capillaries}
\subsection{Intensity profile and ratio}


\section{Nematic liquid crystals in toroids}
\subsection{Measuring the intensity profile and aspect ratio}
\subsection{Large aspect ratio toroids}
\subsection{Small aspect ratio toroids}

\section{Simulating polarized optical microscopy textures for twisted escaped radial director configurations}
\subsection{Jones Calculus}
\subsection{Validation using spherical droplets}
\subsection{Planar-anchored nematic toroids}
\subsection{Comparison with homeoetropic-anchored nematic toroids}
\subsection{Intensity ratio as a function of twist parameter}

\section{Nematic liquid crystals in bent capillaries}
\subsection{Making bent capillaries}
\subsection{Measuring planar curvature}
\subsection{Measuring the intensity profile}
\subsection{Comparison with toroids}

\section{Conclusions}
