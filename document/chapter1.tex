%!TEX root = thesis.tex
\chapter{Introduction}
Ordered materials are are made up of mesogens with a preferred local arrangement that corresponds to a continuous symmetry breaking.
The preferred arrangement can occur at multiple lengthscales and ordered materials make up the majority of the elastic solids.
The local symmetry breaking can manifest itself as a macroscale symmetry that affects the properties of the material.
Example is the birefringence that arises in NLC.
Places where the order is not satisfied (again broken symmetry) are defects.
Defects also affect material properties and are essential to understanding the physics of disordered materials, phase transitions, etc.
Defects can arise from kinetics like grain boundaries in polyxtalline materials.
Defects can arise from energetics like [INSERT EXAMPLE HERE]
Defects can also be required due to the geoemetrical frustration that arises from confinement.
Example is packing on a sphere.
The presence of defects is a global constraint due to the topology of the sphere.
Here I introduce the idea of the Euler characteristic as a topological invariant using the tesselation definition.
There are many ways to satisfy the global constraint, the material chooses the way that minimizes its energy.
Example of scars vs isolated defects.
This is a reflection that locally, defects are a response to the strain on the order caused by the curvature of the confining surface.
Here I introduce curvature and Gaussian curvature and introduce Gauss-Bonnet to relate the integral of K with the Euler characteristic.
Defects act as isolated sources of curvature that interact with the curvature of the surface.
If we now take the spheres and elongate them into rods, the order changes from the 6-fold xtalline order to the 2-fold preferred alignment of nematic order.
Again use the packing on a sphere example of the defects.
Introduce this as a consequence of the Poincare-Hopf theorem where the defect charge is defined as the director rotation about a path encircling the defect.
Example of concentric and bipolar emulsion droplets, the droplet minimizes its bulk free energy under the constraint that the surface satisfies the P-H theorem.
In the absence of the bulk, the surface alone minimizes its energy differently, with 4 $+1/2$ defects.
The curvature sensitivity of defects is not unique to spheres, nematic defects are also sensitive.
Use example that free energy can be written in terms of the Gaussian curvature and the local defect charge.
The interplay between curvature, topology, and order means that studying ordered materials in non-spherical surfaces opens a new world of interesting physics.
Consider the torus.
The Euler characteristic changes so that no defects are required.
However, the postive and negative Gaussian curvature on torus opens up a world of interesting physics.
In fact there are long-standing predictions of defective ground states and defect unbinding for crystals.
The same predictions happened recently for nematics in a toroidal shell as well.
In our lab, we have previously looked at planar-anchored bulk nematic toroids.
While we didn't see defects due to the bulk influence, the varying curvature led to new insights about the role of a previously-neglected distortion called saddle-splay.
The saddle-splay distortion caused a novel twisted state that manifested itself through a spontaneous symmetry breaking reminiscent of magnetism.
While spherical and cylindrical confinement are common, the role of varying curvature and non-spherical topology has not yet been examined.
In this thesis, we extend the study of nematics confined in curved spaces.
We begin with a formal introduction to the theory of NLC.
We then use spherical droplets to uncover the role of curvature in the surface anchoring.
We consider an active nematic on the surface of a torus and gain insights into the role of both activity and curvature on defect dynamics.
We examine equilibrium nematics in a torus with homeotropic anchoring and see how curvature can drive the formation of a twisted state even in the absence of the influence of saddle-splay.
Finally, we return to a spherical topology and investigate nematics confined in a capillary bridge with homeotropic boundary conditions.
We uncover how the curvature of the confining surface can affect the bulk defect state and conformation.
In Chapter 7, we conclude. 
