%!TEX root = thesis.tex
\chapter{Summary and future work}
Confining ordered materials can lead to interesting phenomonology due to the interplay between the order and the confinement geometry.
For the nematic materials used in this Thesis, it is often the curvature of the confining volume or surface that affects the nematic order.
This is a consequence of the fact that nematic materials possess only orientational order with the order defined by the director $\mathbf{n}$; the free energy has order $(\nabla \mathbf{n})^2$, indicating that curvature in the director field costs energy.
The influence of curvature on a nematic material is even more explicit when considering a 2D nematic constrained to lie on a surface.
In this situation, the free energy resembles that of a multicomponent plasma, with defects in the nematic acting like discrete charges in the plasma and the Gaussian curvature of the surface as background charge in the plasma  [see Eq~\ref{e:1-TopTheoryofDefects}].
