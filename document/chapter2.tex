%!TEX root = thesis.tex
\chapter{Fundamentals of nematic liquid crystals}

\section{Local order and defects}
Uniaxial nematic liquid crystals are an ordered phase resulting from breaking the continuous rotation symmetry of a collection of either rod-like or plate-like particles~\cite{RN33}.
In both cases, the shape is anisotropic and characterized by one lengthscale that is different from the other two.
As either the concentration of the particles is increased or the temperature of the system is decreased, the system breaks two continuous rotational symmetries and develops order.
The order is characterized by a preferred direction, where rod-like particles prefer to have their long axis aligned parallel to each other and plate-like particles prefer to have their short axis parallel to each other.
Thus, the system still possesses continuous rotational symmetry about the preferred alignment direction as well as continuous translation symmetry in  all directions.
If the concentration of the particles were further increased of the temperature were further decreased, the particles would break translational symmetry and the nematic phase would transition to a crystalline phase, with three broken continuous translational symmetries.
Hence, the liquid-crystalline nematic phase is an intermediate ``mesophase'' that possesses the continuous translational symmetries of the isotropic ``liquid phase'' as well as some of the broken rotational symmetries of the ``crystalline'' phase.
In the remainder of this chapter and in the rest of the Thesis, we will focus on NLC whose constituent particles are rod-like.


\subsection{The director and the order parameter}
Given a group of rods, we want to determine the phase as well as the alignment direction provided that the rods are in the nematic phase.
Thus, we need an order parameter that goes to $0$ in the isotropic phase, is nonzero in the nematic phase, and contains the preferred alignment direction in the nematic phase.
We will first derive this quantity for a collection of rods laying in a two-dimensional (2D) plane and then generalize to three-dimensions (3D).
Let a collection of rods laying in a 2D plane be indexed by $\alpha = 1,2,\dots, N$, such that the orientation of a given rod can be specified with the unit vector $\mathbf{u}^{\alpha} = u^{\alpha}_i\hat{e}_i$, where we use the Einstein summation convention to sum over repeated indices.
Note that due to the inversion symmetry of the nematic phase $\langle \mathbf{u}^{\alpha}\rangle_{\alpha} = 0$, where $\langle \cdot \rangle_{\alpha}$ represents an ensemble average over all $\alpha$.
Thus, we need a rank-2 object to accomodate the nematic symmetry.
Let:
\begin{equation}
  \mathbf{Q} = \left \langle \mathbf{u}^{\alpha} \otimes \mathbf{u}^{\alpha} - \frac{1}{2} \mathbb{1} \right \rangle_{\alpha},\label{e:2-2DOrderRaw}
\end{equation}
such that $\textrm{Tr}\big \{ \mathbf{Q} \big \} = \langle u^{\alpha}_i u^{\alpha}_i - 1 \rangle_{\alpha} = 0$, making $\mathbf{Q}$ traceless and symmetric.
We need $\mathbf{Q}$ to be traceless if we wish $Q_{ij} = 0$ in the isotropic phase.
Without loss of generality, we can define another arbitrary orthonormal basis in the plane given by $\hat{e}_i'$ such that the transformation from the primed coordinates to the unprimed coordinates is given by $V_{ij} = \hat{e}'_i \cdot \hat{e}_j$.
Note that $\mathbf{V} \cdot \mathbf{V} = V_{ij}V_{ji} = \hat{e}'_i \cdot \hat{e}_j \cdot \hat{e}'_j \cdot \hat{e}_i = \delta_{ij} = \mathbb{1}$, such that $\mathbf{V} = \mathbf{V}^{-1}$, making $\mathbf{V}$ self-adjoint.
Applying $\mathbf{V}$ to $\mathbf{Q}$, we have:
\begin{align}
  \mathbf{V}^T \mathbf{Q} \mathbf{V} &=
  \bigg \langle \big ( \mathbf{V}^T \mathbf{u}^{\alpha} \big ) \otimes \big ( \mathbf{V}^T \mathbf{u}^{\alpha} \big )\bigg \rangle_{\alpha}  - \frac{1}{2} \mathbb{1} \nonumber \\ & =
  \begin{pmatrix}
    \langle \cos^2 \theta^{\alpha}\rangle_{\alpha} - 1/2 & \langle \sin \theta^{\alpha} \cos \theta^{\alpha} \rangle_{\alpha} \\
    \langle \sin \theta^{\alpha} \cos \theta^{\alpha} \rangle_{\alpha} & \langle \sin^2 \theta^{\alpha} \rangle_{\alpha} - 1/2
  \end{pmatrix},\label{e:2-2DOrderRot}
\end{align}
where $\cos \theta^{\alpha} = \mathbf{u}^{\alpha} \cdot \hat{e}_1'$.
Let us now assume that we chose the $\hat{e}'_i$ to be the eigenvectors of $\mathbf{Q}$.
Then $ \mathbf{V}^T \mathbf{Q} \mathbf{V}$ must be diagonal such that $\langle \sin \theta^{\alpha} \cos \theta^{\alpha} \rangle_{\alpha} = \langle \sin (2 \theta^{\alpha}) \rangle_{\alpha} = 0$.
This occurs in two ways.
First, if we choose the collection of rods to be randomly oriented as if we were in the isotropic phase, $\langle \sin (2 \theta^{\alpha}) \rangle_{\alpha} = 0$ as $\sin$ is an odd function.
However, for a randomly oriented set of rods, $\langle \cos^2 \theta^{\alpha}\rangle_{\alpha} - 1/2 = \langle \sin^2 \theta^{\alpha}\rangle_{\alpha} - 1/2 = 0$, such that $\mathbf{Q} = 0$.
Hence, this first method hasn't really diagonalized $\mathbf{Q}$ but instead shown that $\mathbf{Q}$ has the correct behavior in the isotropic phase.
The second way for $\langle \sin (2 \theta^{\alpha}) \rangle_{\alpha} = 0$ is if the collection of rods on average points along $\hat{e}'_i$.
In this scenario, $\theta^{\alpha} \approx 0 \textrm{ or } \pi/2$ such that $\langle \sin (2 \theta^{\alpha}) \rangle_{\alpha} = 0$.
Thus, we see that by diagonalizing Eq.~\ref{e:2-2DOrderRot}, we have determined that the average alignment direction is either along $\hat{e}_1'$ or along $\hat{e}_2'$.
We now take Eq.~\ref{e:2-2DOrderRot} and write it assuming the collection of rods on average points along $\hat{e}'_1$.
This yields:
\begin{equation}
  \mathbf{V}^T \mathbf{Q} \mathbf{V} =
  \begin{pmatrix}
    \langle \cos^2 \theta^{\alpha}\rangle_{\alpha} - 1/2 & 0 \\
    0 & \langle \sin^2 \theta^{\alpha} \rangle_{\alpha} - 1/2
  \end{pmatrix} =
  \begin{pmatrix}
    S/2 & 0 \\
    0 & -S/2
  \end{pmatrix},\label{e:2-2DOrderDiagBig}
\end{equation}
where $S = 2 \langle \cos^2 \theta^{\alpha} \rangle_{\alpha} - 1$ is often called the scalar order parameter and denotes how well-aligned the system is~\cite{RN33}.
For example, if every rod was aligned along $\hat{e}_1'$ such that $\mathbf{u}^{\alpha} = \hat{e}'_1$, $S = 1$.
Similarly, if we again check the isotropic limit of a random collection of rods, we see that $S = 0$, as desired.
Thus, $\mathbf{Q}$ serves as the tensor order parameter for a collection of rods, where the alignment direction corresponds to the eigenvector associated with the largest eigenvalue of $\mathbf{Q}$, and that the eigenvalue itself gives $S$, the scalar order parameter~\cite{RN33}.
We denote this preferred alignment direction in a uniaxial nematic liquid crystalline material with the unit bivector, $\mathbf{n}$, called the director~\cite{RN33}.
In terms of the director, we can now write the diagonalized $\mathbf{Q}$ in 2D as:
\begin{equation}
  \mathbf{Q} = S \left ( \mathbf{n} \otimes \mathbf{n} - \frac{1}{2}\mathbb{1} \right ),\label{e:2-2DOrderDiag}
\end{equation}
where $\theta^{\alpha} = \arccos (\mathbf{u}^{\alpha} \cdot \mathbf{n})$.
Generalizing to 3D, we have a similar expression for a collection of rods~\cite{RN33}:
\begin{equation}
  \mathbf{Q} =  \left \langle \mathbf{u}^{\alpha} \otimes \mathbf{u}^{\alpha} - \frac{1}{3} \mathbb{1} \right \rangle_{\alpha},\label{e:2-3DOrderRaw}
\end{equation}
that when diagonalized becomes:
\begin{equation}
  \mathbf{Q} = S \left ( \mathbf{n} \otimes \mathbf{n} - \frac{1}{3}\mathbb{1} \right ),\label{e:2-3DOrderDiag}
\end{equation}
where in 3D $S = \frac{1}{2} \big \langle 3 \cos^2 \theta^{\alpha}  - 1 \big  \rangle_{\alpha} = \big \langle P_2(\cos \theta^{\alpha}) \big \rangle_{\alpha} $,
where again $\theta^{\alpha} = \arccos (\mathbf{u}^{\alpha} \cdot \mathbf{n})$, and $P_2(\cdot)$ is the $2^{nd}$ Legendre Polynomial~\cite{RN33}.
As in two dimensions, we take the eigenvector associated with the largest eigenvalue to be $\mathbf{n}$.
Thus, for a collection of rods, we can determine the phase and if applicable, $\mathbf{n}$, by calculating $\mathbf{Q}$ according to Eqs.~\ref{e:2-2DOrderRaw} or~\ref{e:2-3DOrderRaw} and diagonalizing.


\subsection{Defects in a nematic}
When a material develops order, there comes the possibility of defects in the order, defined generally as regions where the order is not satisfied.
In NLC, defects are locations where $\mathbf{n}$ is undefined.
These locations can be $0$-dimension ($0$D) point defects, $1$-dimension ($1$D) line defects, or $2$D wall defects. \\

In 2D, defects are characterized by their ``topological charge'', $s$, which is a measure of how much the director rotates along a path encircling the defect.
For a director field parametrized by the angle $\phi(\mathbf{R})$, we calculate the topological charge as:
\begin{equation}
  s = \frac{1}{2 \pi}\oint_{\partial A} \textrm{d}\mathbf{R} \cdot \nabla\phi(\mathbf{R}),\label{eq:2-topCharge}
\end{equation}
where $\partial A$ is the boundary of some area $A$ containing the defect and the integral is performed along the boundary.
Since $\mathbf{n}$ must be continuous on $\partial A$, the symmetry of the nematic phase means that $s \in \frac{1}{2} \mathbb{Z}$, a discrete quantity.
In addition, note that the sign of the defect reflects how the director changes with respect to the path direction.
Explicitly, a counter-clockwise (CCW) $\mathbf{n}$ rotation along a CCW path results in a positive defect while a clockwise (CW) $\mathbf{n}$ rotation along a CCW path results in a negative defect.
Defects are additive, such that $s$ calculated along a path encircling multiple defects will yield a charge that is the sum of the individual charges of the encircled defects.
For example, a path encircling a $s = +1/2$ point defect and a $s = -1/2$ point defect would yield $s_{net} = 0$ and a path encircling a $s = +1$ point defect and a $s = -1/2$ point defect would yield $s_{net}= +1/2$.\\

In 3D, defects are characterized by their ``hedgehog charge'' defined as
\begin{equation}
  q = \frac{1}{4 \pi} \oint_{\partial V}d \theta d \phi \mathbf{n} \cdot \left [ \partial_{\theta} \mathbf{n} \times \partial_{\phi} \mathbf{n} \right ],\label{e:2-hedCharge}
\end{equation}
 where $\theta$ and $\phi$ are the polar and azimuthal spherical angles, respectively, and $\partial V$ is the bounding surface of the closed volume $V$ containing the defect.
The volume $V$ must be topologically like a sphere --- this means that the volume has no no holes not handles such that the Euler characteristic is $\chi = 2$.
Physically, $q$ relates the orientations of $\mathbf{n}$ taken on a surface that is topologically like a sphere enclosing the defect to the number of times the orientations cover the unit sphere.
Note that due to the added dimension, half-integer hedgehog structures are unstable such that $q \in \mathbb{Z}$.
In addition, since there are 2 possible ways to project the apolar nematic orientation to the polar orientations on the unit sphere, any structure in isolation can only be determined up to $|q|$ as the chosen projection will determine the sign.
However, once a projection is chosen, a collection of defects with the same magnitude $|q|$ can be separated by sign into charges with $+q$ and charges with $-q$.
Similar to topological charge in 2D, hedgehog charge is additive such that calculating $q$ in a volume containing a $+q$ point defect and $-q$ point defect charge will yield $q_{net} = 0$. \\

The use of the term ``charge'' to characterize defects in NLC is no accident.
The analogy to electric charges goes beyond the additivity of defect charges --- the analogy also extends to their interactions.
Defects in 2D and 3D with like-signed charge repel and defects with opposite-signed charge attract and even annihilate.




\section{Frank-Oseen free energy}
Since the mesogens in a nematic material prefer to align along the $\mathbf{n}$, the ideal state for a nematic phase is a homogeneously-aligned monodomain with $\mathbf{n}$ a constant everywhere.
Distortions from this uniform state cost energy.
Since in most experiments the director distortions occur over much larger length scales than the molecular length --- $|\nabla \mathbf{n}| a \ll 1$, where $a$ is the molecular length --- we can forget about the behavior of the individual mesogens and instead use a continuum model for the free-energy density that depends on $\mathbf{n}$ only.
Here, we follow the work of F.C. Frank and expand about the undistorted director state in powers of $\nabla \mathbf{n}$.
This is a phenomenological approach similar to Hooke's elasticity theory of a solid; however, instead of focusing on restoring stresses that oppose strains, we look for restoring torques that oppose curvature-strains in the director field.
This is again a reflection that there is no restriction to the center-of-mass positions of the nematic mesogens, the nematic elasticity only opposes deformations in the orientations of the mesogens.

\subsection{A brief derivation}
Let a local coordinate system at a point be defined by $\{x_1, x_2, x_3 \}$ such that we can define $\mathbf{n} = (\mathbf{n} \cdot \hat{e}_i) \hat{e}_i = n_i$, where we again sum over repeated indices, $\hat{e}_i$ is the unit vector associated with $x_i$.
If we let $\hat{e}_3$ be parallel to $\mathbf{n}$ at the point, we can write the $n_i$ as:
\refstepcounter{equation}\label{e:2-LocalCoord}
\begin{align}
  n_1 &= \frac{\partial n_1}{\partial x_1}x_1 + \frac{\partial n_1}{\partial x_2}x_2 + \frac{\partial n_1}{\partial x_3}x_3 + \mathcal{O}\big (x^2 \big ) \nonumber \\
      &= a_1 x_1 + a_2 x_2 + a_3 x_3 + + \mathcal{O}\big (x^2 \big )\tag{\theequation a}\label{e:2-LocalCoordA}  \\
  n_2 &= \frac{\partial n_2}{\partial x_1}x_1 + \frac{\partial n_2}{\partial x_2}x_2 + \frac{\partial n_2}{\partial x_3}x_3 + \mathcal{O}\big (x^2 \big ) \nonumber  \\
      &= a_4 x_1 + a_5 x_2 + a_6 x_3 + + \mathcal{O}\big (x^2 \big )\tag{\theequation b}\label{e:2-LocalCoordB}  \\
  n_3 &= 1 + \mathcal{O}\big (x^2 \big ). \nonumber
\end{align}
Now expanding about the undistorted state, we can write the free-energy density to quadratic order in the first-derivatives as:
\begin{equation}
  f(\mathbf{n}) = K_i a_i + K_{ij} a_i a_j,\label{e:2-FrankGeneralExpansion}
\end{equation}
where $i,j = \{ 1,2 \dots, 6 \}$, giving us 42 possible terms.
However, any free-energy must respect the symmetry of the nematic, hence it must be invariant under exchange, invariant under inversion, invariant under arbitrary rotations about $\mathbf{n}$, and invariant with respect to the handedness of the coordinate system.
Under these conditions, all of the 6 $K_i$ vanish and of the 36 $K_{ij}$, 26 vanish and only 4 are independent, giving the coefficient matrix:
\begin{equation}
  K_{ij} =
  \begin{pmatrix}
    K_{11} & 0 & 0 & 0 & (K_{11}-K_{22}-K_{24}) & 0 \\
    0 & K_{22} & 0 & K_{24} & 0 & 0 \\
    0 & 0 & K_{33} & 0 & 0 & 0 \\
    0 & K_{24} & 0 & K_{22} & 0 & 0 \\
    (K_{11}-K_{22}-K_{24}) & 0 & 0 & 0 & K_{11} & 0 \\
    0 & 0 & 0 & 0 & 0 & K_{33} \\
  \end{pmatrix}.
\end{equation}
Collecting terms, we are left with the expression:
\begin{align}
  f(\mathbf{n}) = \frac{1}{2}K_{11} (a_1 + a_5)^2 + \frac{1}{2}&K_{22} (a_2 - a_4)^2 + \frac{1}{2}K_{33} (a_3^2 + a_6^2) \nonumber \\
    &- (K_{22} + K_{24}) (a_1 a_5 - a_2 a_4).\label{e:2-FrankLocalExpansion}
\end{align}
If we return to the source of the $a_i$ coefficients in Eqs.~\ref{e:2-LocalCoord}(a,b), we can uncover the physical significance of the distortions:
\refstepcounter{equation}\label{e:2-PhysicalDistortions}
\begin{align}
  s_1 = a_1 = \frac{\partial n_1}{\partial x_1} \quad & \quad s_2 = a_5 = \frac{\partial n_2}{\partial x_2} \tag{\theequation a} \\
  t_1 = - a_4 = \frac{\partial n_2}{\partial x_1} \quad & \quad t_2 = a_2 = \frac{\partial n_1}{\partial x_2} \tag{\theequation b} \\
  b_1 =  a_3 = \frac{\partial n_1}{\partial x_3} \quad & \quad b_2 = a_6 = \frac{\partial n_2}{\partial x_3}, \tag{\theequation c}
\end{align}
where we have renamed the coefficients to reflect their associated distortion, with $s_1$ and $s_2$ signifying a ``splay'' distortion, $t_1$ and $t_2$ signifying a ``twist'' distortion, and $b_1$ and $b_2$ signifying a ``bend'' distortion,.
Now if we use Eqs.~\ref{e:2-PhysicalDistortions}(a-c) to re-write Eq.~\ref{e:2-FrankLocalExpansion}, we have:
\begin{align}
  f(\mathbf{n}) = \frac{1}{2}K_{11} (s_1 + s_2)^2 + \frac{1}{2}&K_{22} (t_1 + t_2)^2 + \frac{1}{2}K_{33} (b_1^2 + b_2^2) \nonumber \\
    & - (K_{22} + K_{24}) (s_1 s_2 + t_1 t_2).\label{e:2-FrankPhysicalExpansion}
\end{align}
Hence, we see that each term in the free-energy density comes from a different distortion, such that the elastic constants $K_{ij}$ hold the relative importance of each distortion for a given material.
For example, consider a NLC with $K_{11} \gg K_{33} = K_{22} = K_{24}$ --- under confinement or strain, this material will prefer to bend and twist instead of splay.
Finally, we can move from the local frame and re-cast the $a_i$ in Eq.~\ref{e:2-FrankLocalExpansion} in terms of a single global coordinate system:
\begin{align}
  f(\mathbf{n}) = \frac{1}{2}K_{11} (\nabla \cdot \mathbf{n})^2 + \frac{1}{2}&K_{22} (\mathbf{n} \cdot \nabla \times \mathbf{n})^2 + \frac{1}{2}K_{33} (\mathbf{n} \times \nabla \times \mathbf{n})^2 \nonumber \\
    & - \frac{1}{2}(K_{22} + K_{24}) \nabla \cdot (\mathbf{n}\nabla \cdot \mathbf{n} + \mathbf{n} \times \nabla \times \mathbf{n}),\label{e:2-FrankFinalExpansion}
\end{align}
arriving at the well-known expression for the Frank-Oseen Free Energy.\\

Note that while Frank pioneered the phenomenological approach for a liquid crystalline free-energy, his was not the first attempt to construct a free-energy for liquid crytals.
Oseen arrived at a similar expression to Eq.~\ref{e:2-FrankFinalExpansion} before Frank, but he approached the problem from a microscopic approach under the assumption that the free-energy could be calculated considering interactions between all possible pairs of molecules.
While Oseen's final expression for a nematic misses some of the terms in Eq.~\ref{e:2-FrankFinalExpansion}, it is close enough that Eq.~\ref{e:2-FrankFinalExpansion} carries the names of both Frank and Oseen.


\subsection{Nehring, Saupe, and second derivatives}
We arrived at Eq.~\ref{e:2-FrankFinalExpansion} by expanding about the undistorted state to quadratic order in powers of the 1$^{st}$ derivatives of $\mathbf{n}$.
Due to the symmetry of the nematic, all of the terms linear in $\nabla \mathbf{n}$ vanished, making the quadratic terms the lowest order surviving terms in the expansion.
However, as Nehring and Saupe pointed out, a full expansion to the lowest order should also include 2$^{nd}$ derivatives of $\mathbf{n}$ as they can contribute to order $\mathcal{O}(x^2)$ as well.
In fact, Oseen's original free-energy expression also included terms depending on 2$^{nd}$ derivatives of $\mathbf{n}$ that Frank neglected.
Thus, Eq.~\ref{e:2-FrankGeneralExpansion} becomes:
\begin{equation}
  f(\mathbf{n}) = K_i a_i + K_{ij} a_i a_j + K'_{ij} a_{a,j},\label{e:2-NSGeneralExpansion}
\end{equation}
where $a_{i,j}$ represents $\dfrac{\partial a_i}{\partial x_j}$, giving us 18 terms in the $K'_{ij}$ coefficient matrix.
Again using the symmetries of a nematic, collecting terms, and transforming into a global coordinate system, we arrive at a new expression for the free-energy:
\begin{align}
  f(\mathbf{n}) = \frac{1}{2}&(K_{11} - 2K'_{13}) (\nabla \cdot \mathbf{n})^2 + \frac{1}{2}K_{22} (\mathbf{n} \cdot \nabla \times \mathbf{n})^2 + \frac{1}{2}(K_{33} + 2K'_{13}) (\mathbf{n} \times \nabla \times \mathbf{n})^2 \nonumber \\
    & - \frac{1}{2}(K_{22} + K_{24}) \nabla \cdot (\mathbf{n}\nabla \cdot \mathbf{n} + \mathbf{n} \times \nabla \times \mathbf{n})
      + K'_{13} \nabla \cdot (\mathbf{n} \nabla \cdot \mathbf{n}),\label{e:2-NSFinalExpansion}
\end{align}
where we see that the elastic constants associated with splay and bend are renormalized by $K'_{13}$, and a new term called ``splay-bend'' associated with the distortion $\nabla \cdot (\mathbf{n} \nabla \cdot \mathbf{n})$ has appeared. \\

Since Eqs~\ref{e:2-FrankFinalExpansion} and~\ref{e:2-NSFinalExpansion} are phenomenological expressions, renormalizing the splay and bend elastic constants doesn't affect underlying physics.
This leaves the addition of the splay-bend term as the most significant effect of considering 2$^{nd}$ derivatives of $\mathbf{n}$.
The splay-bend term itself is subject to controversy as its inclusion to the free-energy makes the free-energy impossible to minimize.
This is called the Oldano-Barbero paradox, and stems from the fact that the splay-bend terms introduces boundary conditions on the normal derivative of $\mathbf{n}$.
These excess boundary conditions mean that the 2$^{nd}$-order Euler-Lagrange equations used to minimize Eq.~\ref{e:2-NSFinalExpansion} are in general subject to more than 2 boundary conditions, leading to a free energy that is unbounded from below.
Since the free-energy diverges to minus infinity as the size of the normal derivative of $\mathbf{n}$ increases at the boundary, the Oldano-Barbero paradox leads to a discontinuity in a theory based on small deformations on $\mathbf{n}$.
While there have been attempts to solve the paradox by introducing higher-order terms, or constraining the normal derivatives in the splay-bend term, the most common approach seems to be to neglect the splay-bend term altogether due to its ``unphysical nature''.

\subsection{Insights from microscopic calculations}
There has been significant theoretical effort, starting with the pioneering work of Oseen, to relate the microscopic interactions between the nematic mesogens to the macroscopic, measurable $K_{ij}$ of the nematic phase.
While an in-depth discussion of the various methods and results is beyond the scope of this document, it is worthwhile to examine the general process and consider the implications for the $K_{ij}$ in the phenomenological expressions derived earlier.
At the most basic level, a free-energy density is constructed from the individual mesogens by integrating a pairwise potential over all possible pairs of mesogens, where the potential depends on both relative position and orientation.
Since the interaction potential is positive definite, we see from the microscopics that any elastic constant in Eq~\ref{e:2-FrankFinalExpansion} or Eq.~\ref{e:2-NSFinalExpansion} must be greater than or equal to 0.
However, a free-energy constructed upon the physical mesogens themselves is a many-body problem requiring us to know the position and orientation of each individual mesogen.
Thus, it is common to consider instead the particle density as a function of orientation and position and integrate over all possible points in the volume.
This is a density functional theoretic approach and thus yields a nonlocal free energy similar to the general expression:
\begin{equation}
  \delta F = \int \textrm{d}\mathbf{r} \textrm{d}\mathbf{r}' \, G(\mathbf{r},\mathbf{r}'),
\end{equation}
where $\mathbf{r}$ and $\mathbf{r}'$ are positions, and $G(\mathbf{r},\mathbf{r}') = 0$ for $|\mathbf{r}-\mathbf{r}'| \gg 1$, reflecting the finite interaction range of the mesogens.
However, since we are looking to derive a local expression for the free-energy as a functional of director field, we must satisfy
\begin{equation}
  \delta F = \int \textrm{d}\mathbf{r} \textrm{d}\mathbf{r}' \, G(\mathbf{r},\mathbf{r}')
  = \int \textrm{d} \mathbf{R} \, g(\mathbf{R}),\label{e:2-NonlocalLocal}
\end{equation}
where $g(\mathbf{R})$ is a local function of position, $\mathbf{R}$.
Since there are an infinite number of possibilities to perform the nonlocal-to-local mapping $(\mathbf{r},\mathbf{r}') \rightarrow \mathbf{R}$, any elastic constants in $f(\mathbf{R})$ cannot depend on the choice of mapping.
In addition, the microscopic calculations are performed with a density function that is dependent on position and orientation, $\rho(\mathbf{R},\mathbf{\omega})$, where $\mathbf{\omega}$ specifies an orientation; however, the phenomenological free energy expressions are only a function of the director field.
Thus, we must also need to satisfy:
\begin{equation}
  \delta F[\mathbf{n}] = \int \textrm{d} \mathbf{R} \, f(\mathbf{n}) = \int \textrm{d} \mathbf{R} \, g \big ( \rho(\mathbf{R},\mathbf{\omega}) \big),\label{e:2-LessInfo}
\end{equation}
where $\mathbf{n}$ is a function of $\mathbf{R}$.
Since the director contains less information than the complete density function, in general there is not a unique mapping from  $\rho(\mathbf{R},\mathbf{\omega})$ to $\mathbf{n}(\mathbf{R})$.
To get a unique mapping, we have to restrict the possible distortions to the weakest possible modes in the long-wavelength $(\xi \rightarrow \infty)$ and small amplitude $(\epsilon \ll 1)$ regime.
These modes come from the Nambu-Goldstone theorem, which requires that any broken continuous symmetry results in a soft mode, i.e.\ a mode whose distortion cost vanishes as $\xi^{-2}$ and whose susceptibility diverges in the limit $\xi \rightarrow \infty$.
These soft modes are the weakest modes possible and thus will define the leading order in the free energy, making the free energy a unique functional of $\mathbf{n}$. \\

Given these conditions, microscopic calculations show that $K_{11}$, $K_{22}$, $K_{33}$, and $K_{24}$ are real physical material properties.
However, $K'_{13} = 0$ as it cannot contribute a mapping-independent distortion that vanishes as $\xi^{-2}$.
The conclusion that $K'_{13} = 0$ also holds when calculations are done close to a boundary provided the $\mathbf{n} = -\mathbf{n}$ inversion symmetry is maintained.
Since inversion symmetry is assumed in deriving the phenomenological expressions in Eqs.~\ref{e:2-NSFinalExpansion} and~\ref{e:2-FrankFinalExpansion}, situations that violate inversion symmetry create far more issues than simply the inclusion of a $K_{13}'$ term.
Note that while there have been microscopic calculations that predict a nontrivial $K'_{13}$, these calculations have specified a $(\mathbf{r},\mathbf{r}') \rightarrow \mathbf{R}$ mapping, rendering their $K'_{13}$ values invalid.
Thus, the Oldano-Barbero paradox is resolved and the common approach of neglecting splay-bend is valid.
With $K_{13}'=0$, we see that Eq.~\ref{e:2-NSFinalExpansion} reduces to the expression for the Frank-Oseen free energy in Eq.~\ref{e:2-FrankFinalExpansion}.
Microscopic calculations also show that close to a boundary, $K_{ij}$ in general become position dependent as the molecules near to the boundary now interact with whatever material is outside of the nematic volume.
However, this interaction limited by the range of the intermolecular potential, such that the position dependence of the $K_{ij}$ vanishes over a small boundary layer --- beyond this layer the $K_{ij}$ return to their bulk values that depend only on the specific nematic material.\\


In summary, microscopic calulations show that the free-energy of a NLC is composed of terms quadratic in $\nabla \mathbf{n}$.
Even though terms linear in $\nabla ^2 \mathbf{n}$ could in principle contribute to the free-energy to the same order as terms quadratic in the 1$^{st}$ derivatives, microscopic calculations show that the coefficients associated to 2$^{nd}$ derivative distortions vanish, thus removing the problems that the 2$^{nd}$ derivatives distortions bring
In addition, since the free-energy is constructed using a positive-definite pairwise potential, all the elastic constants in the phenomenological free-energy expressions must be greater than or equal to 0.


\subsection{Saddle-splay and curvature-coupling}
If we take a look at the terms in the Frank-Oseen free-energy in Eq.~\ref{e:2-FrankFinalExpansion}, we can see a difference between the saddle-splay term and the splay, twist, and bend terms.
Notice that since all the $K_{ij} \geq 0$, the energetic cost of splay, bend, and twist distortions are positive semi-definite.
However, notice that the saddle-play distortion can be either positive or negative.
In addition, the saddle-splay distortion is a total differential, giving us the option of re-writing the distortion in terms of a surface integral using Gauss's theorem:
\refstepcounter{equation}\label{e:2-F24Gauss}
\begin{align}
  F_{24}[\mathbf{n}] &= - \frac{1}{2}\int_{V} \textup{d}^3  \mathbf{R} \bigg \{ (K_{22} + K_{24})  \nabla \cdot (\mathbf{n}\nabla \cdot \mathbf{n} + \mathbf{n} \times \nabla \times \mathbf{n}) \bigg \}\tag{\theequation a}  \\ &=
  - \frac{1}{2}(K_{22} + K_{24}) \oint_{\partial V} \textup{d}^2  \mathbf{R} \bigg \{   \mathbf{k} \cdot (\mathbf{n}\nabla \cdot \mathbf{n} + \mathbf{n} \times \nabla \times \mathbf{n}) \bigg \},\tag{\theequation b}
\end{align}
where $V$ is a volume and $\partial V$ is a piecewise-smooth manifold bounding $V$.
Note that the $\mathbf{n}$ needs to be defined everywhere on $\partial V$, meaning that the manifold needs to defect-free.
However, since $\partial V$ only needs to be piecewise continuous, we can always draw the manifold to exclude any defects.
Therefore, Gauss's law can always be applied, implying that any nontrivial saddle-splay distortion in $V$ must affect $\mathbf{n}$ on $\partial V$.
In contrast, nontrivial splay, bend, and twist distortions can exist purely in the bulk, such that $\mathbf{n}$ is affected in $V$ but not on $\partial V$.
In fact, the ability to generate pure bulk distortions for splay, twist, and bend allow for easy measurements of $K_{ij}$ via the Freederickz transitions.
In these measurements, the bulk volume is always large enough such that any contribution from the position-dependence of $K_{ij}$ near the boundaries is negligible. \\

Due to the connection between the bulk saddle-splay distortion and the director orientation on a bounding manifold, the saddle-splay distortion term is often referred to as a surface-like term and sometimes even referred to as an effective anchoring term.
Since any measurement of $K_{24}$ requires $\mathbf{n}$ to change at an interface, there is concern that the position-dependence of $K_{ij}$ near an interface leads to the inability to measure the bulk value of $K_{24}$ without contamination from the interface.
However, we emphasize that while it is tempting to treat $F_{24}$ as a surface term where the energy cost is calculated using the value of $K_{ij}$ at the interface like:
\begin{equation}
  F_{24}[\mathbf{n}] = - \frac{1}{2}(K_{22}(\mathbf{R}) + K_{24}(\mathbf{R}))_{\mathbf{R} \rightarrow \partial V} \oint_{\partial V} \textup{d}^2  \mathbf{R} \bigg \{   \mathbf{k} \cdot (\mathbf{n}\nabla \cdot \mathbf{n} + \mathbf{n} \times \nabla \times \mathbf{n}) \bigg \},
\end{equation}
this is incorrect as Stokes's theorem cannot be used if $K_{ij}$ is position-dependent since the integrand in Eq.~\ref{e:2-F24Gauss}a is no longer a total differential.
In fact, measuring $K_{24}$ is in principle no different than measuring any of the $K_{ii}$ --- provided the volume of the bulk is much larger than the boundary layer, the influence of the position-dependence of $K_{ij}$ is negligible.
To demonstrate this, we return to Eq.~\ref{e:2-F24Gauss}a and insert the position-dependence:
\begin{equation}
  F_{24} = -\frac{1}{2}\int_{V}\textup{d}^3 \left \{\mathbf{R} (K_{22} (\mathbf{R}) + K_{24}(\mathbf{R})) \nabla \cdot (\mathbf{n} \nabla \cdot \mathbf{n} + \mathbf{n} \times \nabla \times \mathbf{n}) \right \},\label{e:2-F24GaussPosition}
\end{equation}
Integrating Eq.~\ref{e:2-F24GaussPosition} by parts with $\mathbf{A}(\mathbf{R}) = \mathbf{n} \nabla \cdot \mathbf{n} + \mathbf{n} \times \nabla \times \mathbf{n}$ and $K_*(\mathbf{R}) = K_{22} (\mathbf{R}) + K_{24}(\mathbf{R})$ yields:
\begin{align}
  \int_{V}\textup{d}^3\mathbf{R} \left \{ K_{*} (\mathbf{R}) \nabla \cdot \mathbf{A} \right \} &=
  \oint_{\partial V}\textup{d}^2 \mathbf{R} \left \{ \mathbf{k} \cdot (K_{*} (\partial V) \mathbf{A}(\partial V) \right \} -  \int_{V} \textup{d}^3\mathbf{R} \left \{ \mathbf{A} \cdot \nabla K_{*} (\mathbf{R}) \right \}  \\ &=
  K_*(\partial V) \oint_{\partial V} \textup{d}^2\mathbf{R} \left \{ \mathbf{k} \cdot \mathbf{A}(\partial V)\right \} - \int_{V} \textup{d}^3\mathbf{R} \left \{ \mathbf{A} \cdot \nabla K_{*} (\mathbf{R}) \right \}, \label{e:2-pos_stokes1}
\end{align}
where we have chosen $\mathbf{k}$ as the outward-pointing unit normal and we have pulled $K_*(\partial V)$ out of the integral as $K_*$ only depends on the distance from the boundary.
Now, if we separate $K_*(\mathbf{R}) = K_*^0 + K_*^s(\mathbf{R})$ into the bulk material constant $K_*^0$ and the position-dependent contribution $K_*^s(\mathbf{R})$, where $K_*^s(\mathbf{R})$ decays across a boundary layer of thickness $d_l$.
Substituting this into Eq.~\ref{e:2-pos_stokes1}, we are left with:
\begin{align}
  \int_{V}\textup{d}^3\mathbf{R} \left \{ K_{*} (\mathbf{R}) \nabla \cdot \mathbf{A} \right \} &=
  (K_*^0+K_*^s(\partial V)) \oint_{\partial V} \textup{d}^2\mathbf{R} \left \{ \mathbf{k} \cdot \mathbf{A}(\partial V)\right \} \nonumber \\&
  \quad \quad - \int_{V} \textup{d}^3\mathbf{R} \left \{ \mathbf{A} \cdot \nabla K_{*}^s (\mathbf{R}) \right \}. \label{e:2-pos_stokes2}
\end{align}
If we now expand $\mathbf{A}(\mathbf{R})$ and $\nabla K_*^s(\mathbf{R})$ over the boundary layer with $\mathbf{R} = \bm{\alpha} - \zeta \mathbf{k}$, where $\bm{\alpha}$ describes the boundary and $\zeta$ the distance from the boundary, we have:
\begin{align}
  \mathbf{A}(\bm{\alpha} - \zeta \mathbf{k}) &\approx \mathbf{A}(\bm{\alpha}) - \zeta (\mathbf{k} \cdot \nabla \mathbf{A}(\bm{\alpha}))_{\zeta = 0} \label{e:2-expandA} \\
  \nabla K_*^s(\bm{\alpha} - \zeta \mathbf{k}) &\approx \mathbf{k} (\mathbf{k} \cdot \nabla K_*^s(\bm{\alpha} - \zeta \mathbf{k}))_{\zeta = 0} - \zeta\, \mathbf{k}(\mathbf{k} \cdot \nabla (\mathbf{k} \cdot \nabla K_*^s(\bm{\alpha} - \zeta \mathbf{k})))_{\zeta = 0}
\end{align}
As $K_*^s(\mathbf{R})$ decays across a boundary layer of thickness $d_l$, we can approximate the leading order term in $\nabla K_*^s(\bm{\alpha} - \zeta \mathbf{k})$ as:
\begin{equation}
  \mathbf{k} (\mathbf{k} \cdot \nabla K_*^s(\bm{\alpha}))_{\zeta = 0} =
  \begin{cases}
    \mathbf{k}\frac{K_*^s(\bm{\alpha})}{d_l} & 0 \leq \zeta \leq d_l \\
    0 & \textup{otherwise}
  \end{cases}
\end{equation}
Thus, to leading order we can write the 2$^\textrm{nd}$ term in Eq.~\ref{e:2-pos_stokes2} as
\begin{align}
  \int_{V} \textup{d}^3\mathbf{R} \left \{ \mathbf{A} \cdot \nabla K_{*}^s (\mathbf{R}) \right \} & \approx
  \int_{V} \textup{d}^3\mathbf{R} \left \{\mathbf{A}(\partial V) \cdot \mathbf{k}(\mathbf{k} \cdot \nabla K_*^s(\partial V))  + \mathcal{O}(\zeta) \right \} \\ & \approx
  \mathbf{k} \cdot \nabla K_*^s(\partial V) \int_{V} \textup{d}^3\mathbf{R}\left \{\mathbf{A}(\partial V) \cdot \mathbf{k} + \mathcal{O}(\zeta) \right \} \\ & \approx
  \frac{K_*^s(\partial V)}{d_l} \oint_{\partial V} \textup{d}^2\mathbf{R}\left \{\mathbf{A}(\partial V) \cdot \mathbf{k} \int\limits_{0}^{d_l} \textup{d}\zeta \left \{ 1 + \mathcal{O}(\zeta) \right \}\right \} \\ & \approx
  K_*^s(\partial V) \oint_{\partial V} \textup{d}^2\mathbf{R}\left \{ \mathbf{k}  \cdot \mathbf{A}(\partial V) \right \} + \mathcal{O}(d_l^2).
\end{align}
Substituting back into Eq.~\ref{e:2-pos_stokes2}, we have:
\begin{equation}
  \int_{V}\textup{d}^3\mathbf{R} \left \{ K_{*} (\mathbf{R}) \nabla \cdot \mathbf{A} \right \} \approx
  K_*^0 \oint_{\partial V} \textup{d}^2\mathbf{R} \left \{ \mathbf{k} \cdot \mathbf{A}(\partial V)\right \} + \mathcal{O}(d_l^2).\label{e:2-positionF24}
\end{equation}
Now that we again have a total differential, we can use Stokes's theorem and write saddle-splay free energy in Eq.~\ref{e:2-positionF24} as:
\begin{equation}
  F_{24} \approx -\frac{1}{2}(K_{22}^0 + K_{24}^0)
  \oint_{\partial V}\textup{d}^2 \mathbf{R} \left \{\mathbf{k} \cdot (\mathbf{n} \nabla \cdot \mathbf{n} + \mathbf{n} \times \nabla \times \mathbf{n}) \right \} + \mathcal{O}(d_l^2). \label{e:2-positionF24_surf}
\end{equation}
Thus, even through the saddle-splay distortion must be measured in the presence of an interface, where $K_{ij}$ are position-dependent, the free-energy of the saddle-splay distortion is primarily driven by the bulk value of $K_{24}$ and $K_{22}$ with corrections on the order of the square of the boundary layer thickness.\\

The saddle-splay distortion can also serve to couple the director to the curvature of the interface.
Consider a nematic constrained to lie in the plane of the interface.
Then, ignoring the corrections in Eq.~\ref{e:2-positionF24_surf}, we can write the free-energy of the saddle-splay distortion as:
\begin{equation}
  F_{24} = -\frac{1}{2}(K_{22} + K_{24})
  \oint_{\partial V}\textup{d}^2 \mathbf{R} \left \{\mathbf{k} \cdot ( \mathbf{n} \times \nabla \times \mathbf{n}) \right \},\label{e:2-K24PlanDegen}
\end{equation}
since $\mathbf{n} \cdot \mathbf{k} = 0$. Now re-arranging the integrand, we can take advantage of the fact that $\mathbf{n} \cdot \mathbf{n}=1$ and write:
\begin{equation}
  \mathbf{k} \cdot ( \mathbf{n} \times \nabla \times \mathbf{n}) = -\mathbf{k} \cdot (\mathbf{n} \cdot \nabla)\mathbf{n}.\label{e:2-K24rearrange1}
\end{equation}
Since $\mathbf{k} \cdot \mathbf{n} = 0$, we can re-write the left-hand side (LHS) of Eq.~\ref{e:2-K24rearrange1} as:
\begin{equation}
  -\mathbf{k} \cdot (\mathbf{n} \cdot \nabla)\mathbf{n} = \mathbf{n} \cdot (\mathbf{n} \cdot \nabla)\mathbf{k} = -\mathbf{n} \cdot \mathbf{L} \cdot \mathbf{n},\label{e:2-K24rearrange2}
\end{equation}
where $\mathbf{L}$ is the Weingarten Matrix as defined in Eq.~[INTRO]\fxnote{ref Intro}.
Note that because $\mathbf{n} \cdot \mathbf{k}=0$, we can re-write $\mathbf{n}$ in terms of an orthonormal basis defined on the surface like $\mathbf{n} = (\hat{e}_i \cdot \mathbf{n})\hat{e}_i$, where now $i = 1,2$ as we have restricted $\mathbf{n}$ to be on the 2D surface.
This lets us write the saddle-splay free energy in terms of the principle curvatures $\kappa_1$ and $\kappa_2$ as:
\begin{equation}
  F_{24} = \frac{1}{2}(K_{22} + K_{24})
  \oint_{\partial V}\textup{d}^2 \mathbf{R} \left \{\kappa_1 n_1^2 + \kappa_2 n_2^2 \right \},\label{e:2-K24SurfCouple}
\end{equation}
where $n_1$ and $n_2$ are the components of the director along the $1^{st}$ and $2^{nd}$ principle curvatures, respectively.
Thus, for $\mathbf{n}$ at an interface constrained to lie in the plane of  the interface, the free-energy of the saddle-splay distortion is minimized when $\mathbf{n}$ at the interface is aligned along most negative curvature.
We emphasize that this curvature-coupling does not result from interactions between the NLC mesogens at the interface and the material outside of the nematic volume --- it comes from the interactions in the nematic volume itself.




\section{Landau-deGennes free energy}
The Frank-Oseen free energy in Eq.~\ref{e:2-FrankFinalExpansion} is not the only phenomenological free energy expression.
In fact, if we wish to understand the nematic-isotropic phase transition, we need to write a phenomenological free energy that depends on $\mathbf{Q}$.
This tensor-based approach to a free-energy was developed by de Gennes in the spirit of a Landau-type theory.
However, the $\mathbf{Q}$-based phenomenological free energy can can also be used to calculate a distortion free-energy by expanding in invariants of $\nabla \mathbf{Q}$, similar to the derivation of the Frank-Oseen free energy.


\subsection{The isotropic-nematic phase transition}
We consider a free-energy density built on an expansion in powers of $\mathbf{Q}(\mathbf{R})$.
Crucially, the symmetry of a Landau-type free energy describing a phase transition needs to be the same as the higher-symmetry phase.
Thus, to describe the isotropic-nematic phase transition we need the free energy to be rotation and translation invariant --- the expansion is not just in powers of $\mathbf{Q}$, but in terms of scalar invariants of powers of $\mathbf{Q}$.
Given that we wish to predict the phase transition, the positional dependence of $\mathbf{Q}(\mathbf{R})$ is unnecessary, so we will instead consider a mean field approximation $\langle \mathbf{Q}(\mathbf{R}) \rangle = \bm{\mathcal{Q}}$, such that we can write $\bm{\mathcal{Q}}$ in 3D without loss of generality as:
\begin{equation}
  \bm{\mathcal{Q}} =
    \begin{pmatrix}
        -\frac{S}{3} & 0 & 0 \\
        0 & -\frac{S}{3} & 0 \\
        0 & 0 & \frac{2S}{3}
    \end{pmatrix}.
\end{equation}
If we consider the trace and determinant of $\bm{\mathcal{Q}}^p$, where $p \in \mathbb{N}$, we see from Table~\ref{t:2-powersQ} that invariants of powers of $\bm{\mathcal{Q}}$ can be written in terms of powers of $S$.
Therefore, we can write the Landau-de Gennes free energy for the phase transition in 3D in the form:
\begin{equation}
  f_{phase}(\bm{\mathcal{Q}}) = f(S) = f_0 + \frac{1}{2}A S^2 + \frac{1}{3}B S^3 + \frac{1}{4}C S^4 + \mathcal{O} \left (S^5 \right ),\label{e:2-LdGTransGeneral}
\end{equation}
where $f_{phase}(S) > 0$ implies the isotropic phase is stable against the nematic phase and $f_{phase}(S) < 0$ corresponds to the nematic phase as the stable phase, with the phase transition occurring at $f_{phase}(S) = 0$.
\begin{table}[t]
  \centering
  \caption{Scalar invariants of powers of $\bm{\mathcal{Q}}$ in 3D}
  \label{t:2-powersQ}
  \begin{tabular}{|l l|}
    \hline
    $\textup{Tr} \big \{ \bm{\mathcal{Q}} \big \} = 0$ & \\
    $\textup{Tr} \big \{ \bm{\mathcal{Q}}^p \big \} \propto S^p,$ & $p = 2,3, \dots$ \\
    $\textup{Det} \big \{ \bm{\mathcal{Q}}^p \big \} \propto S^{3p}$, & $p = 1,2,3,\dots$ \\
    \hline
  \end{tabular}
\end{table}
In principle, the coefficients $A$, $B$, and $C$ are temperature dependent.
However, in practice, often only the coefficient associated with the lowest order term contains temperature dependence.
Specifically, limiting the temperature dependence to only $A$ is also consistent with molecular theories of the nematic-isotropic phase transition.
Thus, let $A = A_0(T-T_{NI})$, $B = B_0$, and $C = C_0$, allowing us to rewrite Eq.~\ref{e:2-LdGTransGeneral} as:
\begin{equation}
  f_{phase}(S) = f_0 + \frac{1}{2}A_0(T-T_{NI}) S^2 + \frac{1}{3}B_0 S^3 + \frac{1}{4}C_0 S^4.\label{e:2-LdGTransFinal}
\end{equation}
We see that Eq.~\ref{e:2-LdGTransFinal} predicts a first-order phase transition provided $B_0 \neq  0$, as a nonvanishing $B_0$ means that at $T = T_{NI}$, the minima in $f_{phase}(S)$ occur at $S = 0$ and $S = -\frac{B}{C}$, giving a discontinuity in $S$ across the transtion.
Indeed, observations that state functions such as the density and $S$ are discontinuous through the nematic-isotropic transition confirm that the nematic-isotropic phase transition in 3D is first-order. \\

Briefly, if we follow the same procedure for the nematic-isotropic phase transition in 2D, we can write $\bm{\mathcal{Q}}$ as:
\begin{equation}
  \bm{\mathcal{Q}} =
  \begin{pmatrix}
    S/2 & 0 \\
    0 & -S/2
  \end{pmatrix},
\end{equation}
such that $\textup{Tr} \big \{ \bm{\mathcal{Q}}^p \big \} = 0$ for $p = 1,3,5,\dots$.
This implies that a Landau-type free-energy in 2D will look like:
\begin{equation}
  f_{phase}(S) = f_0 + \frac{1}{2}A_0(T-T_{NI}) S^2 + \frac{1}{4}C_0 S^4.
\end{equation}
Now with the absence of a term proportional to $S^3$, the phase transition is predicted to be continuous, as $S$ can vary continuously from $S=0$ in the isotropic phase to $S \neq 0$ in the nematic phase as $T$ passes through $T_{NI}$.
This can be seen by setting $T = T_{NI}$ and noticing that there is only one minimum in $f_{phase}(S)$, and it occurs at $S=0$.
Note that experiments of thin films of NLC have yet to show a continuous nematic-isotropic transition; however, the films always have a macroscopic thickness such that no experiment has yet probed the nematic-isotropic phase transition in a truly 2D geometry.
However, the nematic-isotropic phase transition in 2D has been explored heavily in simulations, and while the literature agree that the transition should be continuous, there is no clear consensus on the specific order of the transition.


\subsection{The distortion free-energy}
If we consider $\mathbf{Q}(\mathbf{R})$ instead of the mean-field approximation $\bm{\mathcal{Q}}$, we can write a distortion free-energy similar to the Frank-Oseen free energy.
Since a Landau-type free energy is an expansion near in the order parameter near the phase transition, in general is is only valid near the phase transition.
However, in NLC $S(T_{NI}) \approx 0.3 \textup{ to } 0.4$, and $S(T > T_{NI}) \approx 0.6 \textup{ to } 0.8$, making the Landau-deGennes free energy valid ``deep'' into the nematic phase.
There are 3 independent scalar invariants quadratic in $\nabla\mathbf{Q}$, allowing us to write the distortion free-energy density as:
\begin{equation}
  f_d(\mathbf{Q}) = \frac{1}{2} L_1 \frac{\partial Q_{ij}}{\partial x_k} \frac{\partial Q_{ij}}{\partial x_k}
    + \frac{1}{2} L_2 \frac{\partial Q_{ij}}{\partial x_j} \frac{\partial Q_{ik}}{\partial x_k}
    + \frac{1}{2} L_3 \frac{\partial Q_{ij}}{\partial x_k} \frac{\partial Q_{kj}}{\partial x_i}\label{e:2-LdGDistGeneralQuadOrder}
\end{equation}
Note that the $L_i$ are constants that only depend on the interactions between the molecules and are temperature independent.
The temperature-dependence of the Landau-deGennes distortion free energy is included in $\mathbf{Q}$ through $S$.
Since the Frank-Oseen free energy does not include $S$, the $K_{ij}$ themselves must change with temperature such that the magnitude  of the free energy is temperature dependent.
Thus, reconciling the Frank-Oseen free energy and the Landau-deGennes distortion free energy can yield insights into the temperature dependence of the $K_{ij}$.
To do this, we start by substituting the definition of $\mathbf{Q}$ from Eq.~\ref{e:2-3DOrderDiag} into the terms of Eq.~\ref{e:2-LdGDistGeneralQuadOrder} and simplifying. Using the notation that
    $A_1 = (\nabla \cdot \mathbf{n})^2$,
    $A_2 = (\mathbf{n} \cdot \nabla \times \mathbf{n})^2$,
    $A_3 = (\mathbf{n} \times \nabla \times \mathbf{n})^2$, and
    $A_{24} = \nabla \cdot (\mathbf{n} \nabla \cdot \mathbf{n} + \mathbf{n} \times \nabla \times \mathbf{n})$, we have the relations:
\refstepcounter{equation}\label{e:2-LdGRelationsQuadOrder}
\begin{align}
  \frac{\partial Q_{ij}}{\partial x_k} \frac{\partial Q_{ij}}{\partial x_k} & =
    2 S^2(A_1 + A_2 + A_3 -A_{24})\tag{\theequation a} \\
  \frac{\partial Q_{ij}}{\partial x_j} \frac{\partial Q_{ik}}{\partial x_k} &=
    S^2 (A_1 + A_3)\tag{\theequation b}\\
  \frac{\partial Q_{ij}}{\partial x_k} \frac{\partial Q_{kj}}{\partial x_i} &=
    S^2 (A_1 + A_3 - A_{24})\tag{\theequation b}
\end{align}
Now we can re-write Eq.~\ref{e:2-LdGDistGeneralQuadOrder} using Eqs.~\ref{e:2-LdGRelationsQuadOrder}(a-d) and collect terms like:
\begin{align}
  f_d &= L_1 S^2 (A_1 + A_2 + A_3 -A_{24}) + \frac{1}{2}L_2 S^2 (A_1 + A_3) \nonumber \\
  & \quad \quad + \frac{1}{2}L_3 S^2 (A_1 + A_3 - A_{24}) \\
      &= \frac{1}{2}A_1(2 L_1 S^2 + L_2 S^2 +L_3 S^2) + \frac{1}{2}A_2(2 L_1 S^2) \nonumber \\
      & \quad \quad + \frac{1}{2}A_3(2 L_1 S^2 + L_2 S^2 +L_3 S^2) \nonumber \\
      & \quad \quad - \frac{1}{2}A_{24}(2 L_1 S^2 + L_3 S^2),
\end{align}
allowing us to write the relations:
\refstepcounter{equation}
\begin{align}
  K_{11} &= K_{33} = 2 L_1 S^2 + L_2 S^2 +L_3 S^2\tag{\theequation a}\\
  K_{22} &= 2 L_1 S^2\tag{\theequation b}\\
  K_{24} &= 2 L_1 S^2 + L_3 S^2.\tag{\theequation c}
\end{align}
Unfortunately, note that the Landau-deGennes distortion free energy to quadratic order requires $K_{11}$ and $K_{33}$.
As $K_{11} \neq K_{33}$ generally, it is common to add a single higher-order invariant that contains both $\mathbf{Q}$ and $\nabla \mathbf{Q}$.
While there are 6 possible invariants that could be added, the invariant
\begin{equation}
  Q_{ij}\frac{\partial Q_{kl}}{\partial x_i} \frac{\partial Q_{kl}}{\partial x_j} = S^3 \left [\frac{2}{3}A_3 - \frac{1}{3}(A_1+A_2+A_3-A_{24}) \right ],
\end{equation}
yields the best agreement with experimental data and is thus the most commonly chosen invariant.
We add this invariant to Eq.~\ref{e:2-LdGDistGeneralQuadOrder} to get:
\begin{equation}
  f_d(\mathbf{Q}) = \frac{1}{2} L_1 \frac{\partial Q_{ij}}{\partial x_k} \frac{\partial Q_{ij}}{\partial x_k}
    + \frac{1}{2} L_2 \frac{\partial Q_{ij}}{\partial x_j} \frac{\partial Q_{ik}}{\partial x_k}
    + \frac{1}{2} L_3 \frac{\partial Q_{ij}}{\partial x_k} \frac{\partial Q_{kj}}{\partial x_i}
    + \frac{1}{2} L_4 Q_{ij}\frac{\partial Q_{kl}}{\partial x_i} \frac{\partial Q_{kl}}{\partial x_j},\label{e:2-LdGDistGeneralHighOrder}
\end{equation}
such that we can write the modified relations between the $K_{ij}$ and the $L_i$:
\refstepcounter{equation}
\begin{align}\label{e:2-LdGFrankRelations}
  K_{11} &= 2 L_1 S^2 + L_2 S^2 +L_3 S^2 - \frac{2}{3}L_4 S^3\tag{\theequation a} \\
  K_{22} &= 2 L_1 S^2 - \frac{2}{3}L_4 S^3\tag{\theequation b} \\
  K_{33} &= 2 L_1 S^2 + L_2 S^2 +L_3 S^2 + \frac{4}{3}L_4 S^3\tag{\theequation c}\\
  K_{24} &= 2 L_1 S^2 + L_3 S^2 - \frac{2}{3}L_4 S^3\tag{\theequation d}.
\end{align}


Similar to the approach of Nehring and Saupe with the Frank-Oseen free energy, we can also consider the role of terms linear in $2^{nd}$ derivatives of $\mathbf{Q}$.
The $2^{nd}$ derivative terms predict that both $K_{13}$ and $K_{24}$ could depend linearly on $S$.
However, once $K_{13}$ is required to be zero, all the contributions from the $2^{nd}$ derivative terms vanish.\\


We briefly note that decoupling $S$ from the elastic constants in the Landau-deGennes distortion free energy has practical applications beyond making predictions about the temperature dependence of the Frank-Oseen elastic constants.
Since $\mathbf{n}$ is undefined at a defect, simulating a director field that contains defects requires one to apply exclude every defect using a cutoff length in order to keep the free-energy from diverging.
In addition, it is not trivial to represent to represent a bivector field using rank-1 objects.
In contrast, $\mathbf{Q}$ is a rank-2 tensor such that there is no ambiguity in representing the state of the nematic orientation at any given point.
In addition, since $\mathbf{Q}$ contains S, it can go to 0 at a defect such that no part of the simulation volume needs to be excluded.
This makes the Landau-deGennes distortion free energy the preferred method for numerical simulations of nematic materials.
It is worth noting that a majority of the simulation works ignore the role of $K_{24}$ when making a mapping of $L_i$ to the Frank-Oseen elastic constants.
Thus, since $L_1$ and $L_3$ differ only by the saddle-splay distortion, $L_3$ is typically folded into $L_1$ and the Landau-deGennes free energy is written only in terms of $L_1$, $L_2$, and $L_4$.
However, due to its simplicity and the fact that the $K_{ij}$ are associated with distinct distortions that are easily visualized, the Frank-Oseen free energy is typically used in analytic theory.\\


\subsection{The scaling of the Frank elastic constants}
From the relations in Eqs.~\ref{e:2-LdGFrankRelations}(a-d), we see that to leading order the Frank-Oseen elastic constants scale with $S^2$.
In fact, the $S^3$ scaling comes entirely from the single higher-order term.
Since $S$ is a monotomic function of $T$, the $S$-dependence of $K_{ij}$ also reflects the temperature dependence near the phase transition.
However, as $S(T_{NI})$ is not infinitely small at the phase transition and $S_{T > T_{NI}} \approx S(T_{NI})$, the temperature scaling predicted by the Landau-deGennes free energy can still be used throughout the nematic phase.
In fact, there is work that finds that fitting $L_1$, $L_2$, and $L_4$ from Eqs.~\ref{e:2-LdGFrankRelations}(a-c) to the measured $K_{ii}$ captures qualitatively the behavior of the $K_{ii}$ throughout the nematic range.
Since all the $K_{ii}$ generally scale the same with $S$, it is common to treat their ratio as temperature-independent.
Practically this means that the equilibrium director configuration in a sample is also temperature-independent, even if the actual magnitude of the Frank-Oseen free energy does vary with temperature.




\section{Experimental characterization of nematic liquid crystals}
\subsection{Confinement and boundary conditions}
Successfully confining nematic materials is more than simply forcing the nematic into an arbitrary volume, we also have to specify the boundary conditions.
For example, confining NLC to a sphere where the material is free to take any orientation on the boundary is uninteresting as the director field can remain homogeneous, as if the sphere did not exist.
However, if we enforce homeotropic boundary conditions such that the director must be everywhere perpendicular to the surface, then we see that their must be a defect in the volume.
In fact, according to Eq.~\ref{e:2-hedCharge}, we see that the sum of all the defects in a spherical volume must have $|q_{net}| = 1$.
Similarly, if we instead enforce degenerate planar boundary conditions such that the director must everywhere lie parallel to the surface, the Poincar\'e-Hopf Theorem requires a total topological charge of $s = +2$ on the surface.
Formally, if $\mathbf{k}$ be the boundary normal, homeotropic anchoring has $|\mathbf{k} \cdot \mathbf{n}| = 1$ and degenerate planar anchoring has $|\mathbf{k} \cdot \mathbf{n}| = 0$, where $\mathbf{n}$ is the director at the boundary.
These are not the only two options, we can also have degenerate tilt boundary conditions, where $0< |\mathbf{k} \cdot \mathbf{n}| < 1$, with a tilt angle given by $\theta = \arccos |\mathbf{k} \cdot \mathbf{n}| $.
Note that planar anchoring does not have to be degenerate.
The direction in the plane of the boundary can be specified as well, breaking the degeneracy such that $\mathbf{k} \times \mathbf{n} = \bm{\sigma}$, where $\bm{\sigma}$ is a unit vector.\\

In an experiment, confinement take place either with either solid boundaries, such as glass cells and capillaries, or fluid boundaries such as in emulsions of a nematic material dispersed in an outer immiscible liquid phase.
It is also possible to confine a nematic using both solid boundaries and fluid boundaries, such as in a sessile drop.
In a sessile drop, the solid boundary provides a flat base for the NLC;\@ however, the remainder of the confinement is provided by the free surface, where the NLC is contact with air.
In general, NLC in contact with an arbitrary boundary will be subject to degenerate tilt boundary conditions.
Thus, for experiments that require a specific boundary condition we must treat the boundary.\\

In order to discuss specific strategies for enforcing anchoring, we must first narrow down the compounds we are dealing with.
NLC can be divided into two broad categories, thermotropics and lyotropics.
Thermotropic phases are determined only by their temperature.
If the temperature is too high, the nematic phase will melt to the isotropic phase and if the temperature is too low, the nematic phase will crystallize.
Lyotropic phases are sensitive to concentration as well as temperature.
They are generally composed of a solid dispersed in a solvent such that the suspended solid is free to flow and orient as it chooses.
In this Thesis, we will primarily use 4-Cyano-4'-pentylbiphenyl (5CB), a thermotropic liquid crystal whose mesogens are $\sim 2$ nm long.\@
5CB belongs to the cyanobiphenyl family of NLC.
The cyanobiphynyls are thermotropic materials characterized by a cyano (CN) group  bonded to a biphenyl (C$_6$H$_4$)(C$_6$H$_4$) group bonded to an alkyl group of given length, C$_p$H$_{2p+1}$, where $p$ is the number of Carbon atoms in the group.
In the case of 5CB, we can write the chemical formula as: CN(C$_6$H$_4$)(C$_6$H$_4$)C$_5$H$_{11}$.
Thus, we see that the ``5'' in 5CB refers to the number of Carbon atoms in the alkyl group.
Since the different cyanobiphenyls generally only differ in the number of Carbon atoms in the alkyl group, the anchoring techniques we cover below for 5CB are general and should apply for any pCB. \\

To enforce homeotropic anchoring for 5CB against a smooth, isotropic solid boundary such as a glass capillary or glass slide, it is common to either deposit surfactant molecules or bond silanes to the surface.
In both cases, the smooth surface becomes decorated with long chains that stick up normal to the surface.
These nematic mesogens at the surface are aligned by the chains, enforcing homeotropic anchoring at the surface.
For example, to enforce homeotropic anchoring for 5CB on glass, we dip coat glass slides in a $0.1$\% w/w lecithin in hexane solution and then let the slides dry.
When the hexane evaporates, the polar head of the lecithin molecule is attached to the glass surface, leaving the long tails sticking up from the glass.\\

Similarly, it is common to enforce homeotropic anchoring in solution again using surfactants.
Here, for an oily NLC dispersed in an aqueous phase, the polar head of the surfactant sits in the aqueous phase while the hydrophobic tail inserts itself into the NLC volume, aligning the NLC at the interface.
For example, we use Sodium Dodecyl Sulfate (SDS) in water to enforce homeotropic anchoring in 5CB emulsions and liquid bridges.
While homeotropic anchoring can be enforced with as little as 1 mM SDS in H$_2$O, we use a solution of 8 mM SDS in H$_2$O for the strongest possible anchoring.
At concentrations above 8 mM, SDS forms micelles in H$_2$O such that working with higher concentrations does not yield any more free SDS in the solution that could potentially adsorb to the interface and contribute to a larger anchoring strength.
This concentration is known as the critical micelle concentration (CMC) and is a common feature in solutions of surfactants or other amphiphilic molecules. \\

While smooth, isotropic solid boundaries can exhibit degenerate planar anchoring, in general the anchoring will be degenerate and tilted.
Thus, surface is often coated with a polymer such that the polymer orientationson the surface are random.
For example, spin-coating poly(vinyl alcohol) (PVA) or poly(ethylene oxide) (PEO) onto glass yields degenerate planar anchoring for 5CB.
If a preferred anchoring direction in the plane is desired, the polymer coating can be rubbed in the desired direction, creating ``grooves'' in the polymer coating along the rubbing direction.
The NLC will the align along the rubbing direction, breaking the degeneracy in the anchoring.
Note that rubbed polymer surfaces do not have perfect planar anchoring but rather exhibit a small tilt angle.
Thus, when designing planar-aligned liquid crystalline cells, where the NLC is confined between two parallel plates with planar anchoring, the rubbing direction on the plates will be anti-parallel.
The anti-parallel rubbing aligns the tilt angles at the plates such that the NLC can still form a homogeneously-aligned domain between the plates. \\

For dispersions of oily NLC such as 5CB in a continuous aqueous phase, we add a polymer like PEO or PVA to the aqueous phase.
The polymer adsorbs to the interface between the NLC and the continuous phase, giving degenerate planar anchoring.
We know of no easy way to break the planar degeneracy in the anchoring when dealing with a liquid-liquid interface.
In addition, we note that even though 5CB exhibits degenerate planar anchoring at the interface with pure water, the addition of a polymer strengthens the anchoring as well as acting to stabilize the emulsion when 5Cb is dispersed in H$_2$O. \\




\subsection{Birefringence and optically polarized microscopy}
Once the nematic is confined in the desired volume with the desired boundary conditions, the most common way to study the sample is with optical polarized microscopy.
This technique takes advantage of the birefringence of the NLC to determine the director in the sample.
For a uniaxial nematic, the index of refraction along $\mathbf{n}$ is known as the ``extraordinary index of refraction'', denoted $n_E$, and the index of refraction perpendicular to $\mathbf{n}$ is called the ``ordinary index of refraction'', denoted as $n_o$.
The orientation of $n_E$ is known as the optic axis and in the case of a nematic material corresponds to $\mathbf{n}$.
The birefringence is characterized by $\delta n = n_E-n_o$, the difference in the indices of refraction in the material.
If $n_E > n_o$, $\delta n > 0$ and the material has positive birefringence.
Conversely, if $\delta n < 0$, the material has negative birefringence.
Thus, the index of refraction affecting a light ray incident upon a birefringent material will depend on the polarization of the light ray.
While unpolarized light will be unaffected by the birefringent material, in general any ray of polarized light will have its polarization state affected by the light ray.
Consider linearly polarized light incident upon a birefringent material with $n_E$ along $\hat{x}$ and $n_o$ along $\hat{y}$.
If the incident light is polarized along $\hat{x}$ or $\hat{y}$, the light ray will see only either the extraordinary index of refraction of the ordinary index of refraction and will leave the sample with no change in its polarization state.
However, now let the incident light have a wavelength $\lambda$ and be linearly polarized at $45^o$ such that the incident ray can be decomposed into one component along $\hat{x}$ and another along $\hat{y}$.
Since each component will see a different index of refraction, they will propagate at different speeds, creating a phase difference between the two components of the incident ray.
This phase difference is known as the retardation and in terms of multiples of $2 \pi$ by $\Gamma = \frac{2 \pi \delta n d}{\lambda}$, where $d$ is the thickness of the birefringent material.
For our example of incident light linearly polarized at $45^o$, let $\delta n d = \lambda / 2$ such that $\Gamma =  pi$.
The light after leaving the sample will now be polarized at $-45^o$, and the birefringent material has the properties of a half-waveplate.
Similarly, if $\delta n d = \lambda$, $\Gamma =  2 pi$, the birefringent material is a full-waveplate and the output light will be unchanged.\\

Optical polarized microscopy (OPM) takes advantage of birefringence and turns the change in polarization into changes in transmitted intensity.
The typical setup adds a pair of linear polarizers into the light path of a standard wide-field optical microscope.
A linear polarizer is an optical element that only passes light polarized along a specific axis known as the ``pass axis''.
Thus, unpolarized light incident upon a linear polarizer will have its intensity reduced by half and the output light will be linearly polarized along the pass axis of the polarizer.
Illustrating the light path for optical polarized microscopy, the incident light passes through the first linear polarizer, known as the ``polarizer'', through the sample, through the objective, and through the second linear polarizer, known as the ``analyzer'', before it is incident on the eyepiece or the camera used to capture the microscope image.
If the polarizer and analyzer are ``crossed'', or oriented with their pass axes orthogonal to each other, any isotropic sample will be entirely dark.
This is because the sample does not affect the polarization of the light such that the linearly polarized light from the polarizer cannot pass through the analyzer.
However, if the sample is birefringent, the polarization state of the incident light is changed and some of the light will pass through the analyzer and be transmitted onto the eyepiece or camera.
The intensity pattern in the image output from an OPM setup can then be used to deduce information about the sample.
While for our purposes we typically care about the spatial variation of the optic axis, and thus the spatial variation in $\mathbf{n}$, OPM can also be used to determine the birefringence of the sample.
For example, we consider 5CB confined to spherical nematic droplets.
Under homotropic anchoring the droplet will have the classic ``radial'' director field while under degernate planar anchoring the droplet will have the classic ``bipolar'' configuration.
These configurations can be distinguished by their OPM textures.
