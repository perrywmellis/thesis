%!TEX root = thesis.tex
\chapter{Summary and future work}
Confining ordered materials can lead to interesting phenomonology due to the interplay between the order and the confinement geometry.
For the nematic materials used in this Thesis, it is often the curvature of the confining volume or surface that affects the nematic order.
This is a consequence of the fact that nematic materials possess only orientational order with the order defined by the director $\mathbf{n}$; the free energy has order $(\nabla \mathbf{n})^2$, indicating that curvature in the director field costs energy.
The influence of curvature on a nematic material is even more explicit when considering a 2D nematic constrained to lie on a surface.
In this situation, the free energy can be written to resemble that of a multicomponent plasma, with defects in the nematic acting like discrete charges in the plasma and the Gaussian curvature of the surface as background charge in the plasma  [see Eq~\ref{e:1-TopTheoryofDefects}].
This, if the surface has both positive and negative Guassian curvature and the nematic has both positive and negative defects, the defects are predicted to segregate, with the positive defects migrating to the regions of positive Gaussian curvature and vice versa.

We explore this situation experimentally using an active polymeric nematic depleted to the surface of a toroidal droplet.
Due to the activity, the nematic is filled with pairs of constantly creating and annihilating $s = \pm 1/2$ defects, with the $s = +1/2$ defects acting as self-propelled particles driving the nematic into a turbulent state.
We measure the time-averaged topological charge in regions on the toroidal droplet and find that the average charge varies linearly with the integrated Gaussian curvature in the region.
The slope of this relationship is positive, indicating that our system exhibits defect unbinding.
In contrast to predictions for a system at equilibrium, we find that the active unbinding depends only on the local geometry and is insensitive to the size and aspect ratio of our toroidal droplets.
Comparing our experimental results to a numerical integration of the equations of motion of active nematic defects further illustrates that the defect unbinding also depends on the defect number density and that the unbinding can even be suppressed in the limit of high activity.
Finally, by using topological defects as micro-rheological tracers and quantitatively comparing our experimental and theoretical results, we are able to estimate the Frank elastic constant, the active stress, and the defect mobility of a microtubule-kinesin active NLC.

Overall, our results not only confirm the theory of topological defects on curved surfaces, but also demonstrate how adding activity to an ordered material changes and enriches equilibrium expectations.
For example, because the active unbinding is driven solely by local interactions, we see that a combination of activity and curvature can be used to guide defects in ordered materials.
In addition, our work introduces a new avenue for the quantitative mechanical characterization of active fluids.

So far we have only examined the behavior of the average topological charge $\overbar{s}_{\Theta} = (\overbar{N}^{+}_{\Theta} - \overbar{N}^{-}_{\Theta})/2$, with $\overbar{N}^{\pm}_{\Theta}$ the time-averaged number of $s = \pm1/2$ defects in a region $\Theta$ on the active nematic toroid.
This is only one of the things that our setup can explore.
For example, we have preliminary data showing the time-averaged number density $\overbar{N}_{\Theta}/A_{\Theta} = (\overbar{N}^{+}_{\Theta} + \overbar{N}^{-}_{\Theta})/A_{\Theta}$ depends on not only the local curvature, but also the aspect ratio of the toroid.
It is not clear why the topological charge depends only on local interactions while the defect number density depends on the size and shape of the toroidal droplet.
In addition, there is work with this active nematic in flat space showing that the $s = +1/2$ defects can themselves assemble to form a nematic phase, where $S$ associated to this higher-order nematic phase grows as $\overbar{N}_{\Theta}$ grows.
However, our preliminary data suggest that on a toroid, the $s = +1/2$ defects do not form a nematic phase but instead a polar phase, and that the strength of this polar phase grows as the $\overbar{N}_{\Theta}$ decreases.
These are just two examples of future directions for active nematics in curved spaces; lkjoil
