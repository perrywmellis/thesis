%!TEX root = thesis.tex
\chapter{Introduction}\label{c:1}
At high temperature, an isolated fluid of spherical particles is a uniform system, i.e.\ the probability of finding a particle at a specific position is constant and only depends on the density of the particles.
As this system is invariant under all possible rotations and translations, the system has complete continuous translational and rotational symmetry.
If we lower the temperature, the particle collection will eventually develop order, where the individual particles have a preferred local arrangement.
For example, a crystalline phase has particles in periodic locations and thus is invariant only with respect to a discrete set of translations and rotations.
Thus the ``ordered'' phase has lower symmetry than the isotropic phase as the continuous symmetry of the isotropic phase was broken to achieve the ordered phase.
The advent of order as a result of continuous symmetry breaking is general and is a hallmark of transitions between a multitude of different phases.
For example, in three dimensions the isotropic-to-crystalline phase transition breaks continuous translational and rotational symmetry in all directions; the isotropic-to-smectic phase transition breaks continuous translation symmetry in one direction and continuous rotational symmetry in two directions; and the isotropic-to-nematic phase transition breaks no continuous translational symmetries, only breaking continuous rotational symmetry in at least two directions~\cite{RN175}.
With the development of order comes the rigidity needed to maintain that order.
This rigidity is not isotropic, but reflects the broken symmetries of the phase.
For example, crystalline and smectic materials do not flow easily due to their broken translational symmetries.
In contrast, nematics with their continuous translational symmetries flow far easier as the mesogens only need to maintain their orientation and not their position.
Similarly, a material with both a smectic phase and a crystalline phase will flow easier when it is in the smectic phase compared to when it is in the crystalline phase, reflecting the higher translational symmetry and thus smaller rigidity of the smectic phase.
This phenomenon is not limited to the rigidity; in general, the properties of a phase will reflect its symmetries.
For example, ordered materials often exhibit birefringence, where the index of refraction of the material depends on the polarization direction of the light incident on the material~\cite{RN175}.

Ordered materials can also have defects, defined generally as locations in the material where the preferred local arrangement is not satisfied.
Belying their name, defects are not necessarily undesireable; in fact, they can have important consequences for the physics of the phase.
For example, joining two crystalline domains of incompatible orientations results in defects forming a grain boundary between the two domains.
In a crystallline material, these grain boundaries affect the moduli of the material, and are even responsible for the phenomenon of ``work hardening,'' where plastically deforming and ``working'' the material strengthens the shear modulus.
The plastic deformations cause defects and grain boundaries to proliferate within the material; however, as the defect density rises, so does the energy required to generate a new defect, resulting in the increased resistance to deformation reflected in the strengthened shear modulus.
In addition, defects can also directly mediate phase transitions, as in the celebrated KTHNY theory of melting in 2D~\cite{RN161,RN163}.
Here, the increase in symmetry when a 2D crystalline phase melts to the isotropic phase is a two-step process.
First, pairs of disclinations~\cite{RN61,RN203}, or rotational defects, proliferate, driving the crystalline phase into a hexatic intermediate phase.
Second, these disclination pairs unbind, transforming the hexatic phase into the isotropic phase.
Furthermore, in soft matter, defects have been used to assemble hierarchical structures via preferential chemical linking as well as through long-range defect-defect interactions~\cite{RN43,RN50,RN150,RN157}.

Apart from their effect on physical systems, defects are fascinating objects in their own right as they are extremely sensitive to the intrinsic geometry of the space they inhabit.
Consider, for example, packing rods on a plane.
In order to maximize the volume fraction of the rods, the rods need to align along the same direction, breaking continuous rotational symmetry and yielding the the 2-fold order of the uniaxial nematic phase~\cite{RN204}.
We can characterize this preferred local alignment with a director, $\mathbf{n}$, where $\mathbf{n} = \mathbf{-n}$.
Clearly, it is easy to fill space on the plane with a homogeneous director field.
However, if we now try to pack rods on the surface of a sphere, for example along either the latitude or longitude lines on the Earth's globe, we see that there are defects in the order that correspond to the singular points at the poles, where $\mathbf{n}$ is undefined.
Since you ``cannot comb a hairy ball'', the presence of singularities is no accident --- it is a consequence of confining the nematic to the surface with the topology of a sphere~\cite{RN209,RN169}.

To formally relate the topology of the sphere with the presence of singularities, we first briefly introduce the necessary topological notions.
The sphere is an example of a differentiable surface that is compact, without boundary, and orientable.
Compact surfaces must both be bounded and contain their limit points; the boundary of a surface is defined as the set of points that can be approached from both within and outside the surface; and orientable surfaces have a defined normal vector everywhere.
For example, 2D Euclidean space, $\mathbb{R}_2$ is a surface but it is not bounded and thus is not compact.
Similarly, the 2D open unit disc $\mathbf{r} < d$ is not compact since it does not contain the circle with radius $d$.
However, the 2D unit disc $\mathbf{r} \leq d$ satisfies both conditions and thus is compact.
In addition, we see that $\mathbf{r} \leq d$ has a boundary $\partial \mathbf{R}$ at $|\mathbf{r}| = d$, as $\partial \mathbf{R}$ can be approached by points both within and outside the disc.
We call compact surfaces without a boundary closed surfaces.
Finally, all of these examples are also orientable; it is trivial to define the surface normal everywhere.
For a closed surface, we can write the Gauss-Bonnet theorem like,
\begin{equation}
  \chi = \mathbf{R} = 2(1-g),\label{e:1-GB}
\end{equation}
where $g$ is the genus, or number of handles, of the surface, and $\chi$ is the Euler characteristic of the surface.
The Euler characteristic is a topological invariant --- continuously deforming a surface, i.e. under a homeomorphism, does not change the Euler characteristic.
This implies that any surface with $g=0$ is topologically equivalent, or homoemorphic, to the sphere and thus also has $\chi=2$.
Thus, it is perhaps more appropriate to say, ``you cannot comb a hairy surface with $\chi=2$,'' such that packing rods on any handle-less closed surface will necessarily result in a director field that has singularities.

These singularities are disclinations; we characterize disclinations with their topological charge, measured by quantifying the director rotation along a closed path encircling the disclination~\cite{RN23,RN153,RN203}, defined as:
\begin{equation}
  s = \frac{1}{2 \pi}\oint_{\partial A} \textrm{d}\mathbf{R} \cdot \nabla\phi(\mathbf{R}),\label{eq:1-topCharge}
\end{equation}
with $\partial A$ giving the contour and the angle $\phi(\mathbf{R})$ parameterizing the director field.
If $\phi(\partial A)$ represents the director orientations along the contour enclosing the disclination, we then see that the topological charge of the disclination is the winding number of $\phi(\partial A)$ on the unit circle, $\mathbb{S}_1$.
Hence, we see that regardless of aligning rods along the latitudes or the longitudes on the Earth's globe, the director along a contour encircling either pole rotates by $2 \pi$ such that both poles have charge $s = +1$, bringing the total charge on the surface to $+2$.
A formal statement connecting a vector field on a closed surface with the topology of the surface is given by the Poincar\'e-Hopf theorem:
\begin{equation}
  \sum\limits_i s_i = \chi,\label{e:1-PH}
\end{equation}
where $\chi$ is the Euler characteristic and $s_i$ is the topological charge of the $i^{\rm th}$ singularity, as defined in Eq.~\ref{eq:1-topCharge}.
Importantly, we now see that Eq.~\ref{eq:1-topCharge} is not limited to disclinations in director fields, but can be generally applied to any vector field.
Note then, that the symmetry of the field on the surface dictates the minimum topological charge of any singularity as the field needs to continuous everywhere along $\partial A$.
For nematic director field where $\mathbf{n} = -\mathbf{n}$, $s \in \mathbf{Z}/2$, while for ordinary polar vector fields, $s \in \mathbf{Z}$.
In addition, the Poincar\'e-Hopf Theorem only constrains the total charge on the surface, such that we could construct a director field on a topologically spherical surface with any combination of defects so long as the total topological charge was equal to $+2$.
Physical ordered systems constrained to a closed surface thus must minimize their free energy with respect to the constraint imposed by the topology.

intro curvature here, connect $\chi$ with K

topological theory of defects, finishe with sphere example


For example, as it was predicted~\cite{RN42,RN104,RN43}, such a physical system minimizes its energy not with$2$ $s=+1$ defects but with $4$ $s=+1/2$ defects arranged on the vertices of a tetrahedron.

this situation was experimentally addressed in our group using shells of nematic liquid crystal.

drops made with a microfluidic device

osmotic pressure used to make shit thin

 and then experimentally observed using a thin shell of nematic liquid crystal~\cite{RN105}

however thick shells are interesting as well since the system is no longer 2D, but there is still a topologiacal constraint on the surface

when bulk is availible, singular lines cost energy


can minimize this energy by nucleating boojums to get rid of singular limie

in bulk drop limit we see bipolar drop

new para, return to the topological  theory of defects ... since K is constant it's not actually that interesting

what happens ifwe want varying curvature

one way to gaurentet this is to go to a torus with a handle

we did this already, but we made the torus from a bulk LC.  Again saw no defects.

but did see interetesting K24

confinement it ineteresting even w/o defects

pppp

up to this point we have considered surface constraints on a surface and considered how surface constraints can metter even when there is a bulk or no defects


we can also consider constaints on a bulk.

once more consider sphere, now homeotripc

papp

in this thesis


active nematic is an experimental quirk but is also interesting in own right

etc.




The connection between defects and the space they inhabit is predicted to go beyond topology.
The Gaussian curvature of a surface is the product of the principal curvatures $\kappa_1$ and $\kappa_2$ of the surface at every point.
For
$
 \frac{1}{2\pi} \int K\, \textrm{d}^2$
The theory of defects on curved surfaces connects the topology of the confining surface to the free energy associated to a given defect configuration via the curvature of the surface.
This is a reflection that locally, defects are a response to the strain on the order caused by the confining surface, i.e. ``geometrical frustration''.
This strain is due to curvature, which I introduce and define via the Weingarten matrix.
Now I connect the integral of Gaussian curvature with the Euler characteristic through Gauss-Bonnet to relate the local influence of curvature with the global constraint of topology.
Defects act as isolated sources of Gaussian curvature that are predicted interact with the Gaussian curvature of the surface.
Motivate how spheres and flat spaces are not enough, need to go to spaces with both positive and negative Gaussian curvature.
The easiest space to achieve this is a torus, where the outer portion of the torus has positive Gaussian curvature and the inner portion has negative Gaussian curvature.
Since the torus has a handle, the Euler Characteristic is now 0; this means that the total integrated Gaussian curvature must also be 0 and any director field on the surface of the torus must have net charge 0.
We have done previous work as investigated NLC in a toroid, where the system is now 3D and the bulk matters.
While the system satisfied the topological constraint with no defects, the varying curvature still led to new insights about the role of a previously-neglected distortion called saddle-splay that couples the director orientation in the plane of the surface to the surface curvature.
The saddle-splay distortion caused a novel twisted state that manifested itself through a spontaneous symmetry breaking reminiscent of magnetism.
This is a reflection of how confinement can affect a system, even in the absence of defects.
In addition to surface constraints, confinement can impose bulk constraints.
If we consider again a sphere filled with a nematic such that the director is perpendicular to the boundary, we see that again, there is an irreducible singularity in the bulk.
These bulk singularities are known as hedgehogs and are characterized by their hedgehog charge [[equation here]].
The hedgehog charge measures the number of times the director orientations on a topologically spherical surface enclosing the defect mapped onto the unit sphere cover the unit sphere.
Thus, we see that the director field for a sphere filled with NLC under hometropic boundary conditions must have total hedgehog charge $|d| = 1$, where again, the interplay between the material and the confining surface determines how to satisfy the topological constraint.
Typically in soft condensed systems, temperature is the driving influence causing the material to equilibrate.
However, the recent advent of active materials, where the individual mesogens of the material have their own source of internal energy, has provided opportunities to study non-equilibrium physics where temperature takes a back seat.
When defects are combined with these force-generating active materials, the ordered phase is driven out-of-equilibrium and ``comes to life'', giving rise to a variety of self-regulated behaviors, such as self-sustained oscillations, spontaneous formation of morphological features, and undirected mobility.
In this Thesis, we investigate the role of geometry on the interplay between order and confinement, both in scenarios where temperature drives the system to the ground state and in scenarios where activity brings about non-equilibrium dynamics.
In Chapter 2, we begin with an introduction to the theory of nematic liquid crystals.
Then, in Chapter 3, we consider a two-dimensional active nematic on the surface of a torus and gain insights into the role of both activity and curvature on defect dynamics.
In Chapter 4, we return to equilibrium nematics and investigate a NLC toroid with hometropic boundary conditions and see how curvature can drive the formation of a twisted state even in the absence of the influence of saddle-splay.
Next, in Chapter 5, we use a NLC capillary bridge with homeotropic anchoring to study the influence of confinement shape on defect type.
Finally, in Chapter 6, we use toroids and spherical droplets to uncover the role of the curvature of the confining interface on the anchoring of NLC.
In Chapter 7, we conclude.
