%!TEX root = thesis.tex
\chapter{Active nematics on the surface of a torus}\label{c:3}
\section{Introduction}
Active materials are composed of mesogens that each can convert stored internal energy or ambient energy to kinetic energy, thus driving the material out-of equilibrium~\cite{RN237,RN238,RN40}.
Importantly, this is distinct from global driving mechanisms such as the application of a shear or the imposition of a field; in active materials, the material is driven out-of-equilibrium due to the intrinsic behavior of the constituent particles.
Thus, the materials cannot be understood within the framework of equilibrium statistical mechanics, as the individual particles are subject to forces independent of thermal fluctuations.
Since activity applies to each individual mesogen, the framework of ``active matter'' has been used to study self-organization and self-driven behavior in a variety of systems from flocks of starlings~\cite{RN239,RN240}, collections of robots~\cite{RN241}, colonies of fire ants~\cite{RN242}, cell growth and migration~\cite{RN51,RN160}, and self-propelled particles~\cite{RN168,RN38}.
Note that these examples of active matter span 6 orders of magnitude in size, with behavior such as flocking, giant number fluctuations, chaotic flows, and low-Reynolds number turbulence driven entirely by a single control parameter~\cite{RN237,RN238,RN40}. \\

Much like equilibrium nematics, active nematic materials are composed of individual mesogens that are anisotropic and thus can possess a nematic phase; however, the addition of activity to nematic order brings about an additional contribution to the interaction between the mesogens.
We note that with the exception of bacteria introduced in lyotropic liquid crystals~\cite{RN86}, to date, the majority of experimental and theoretical work on active nematics takes place in 2D.
Thus, to highlight the effect of activity on the nematic mesogens, we consider two active rod-like particles in 2D.
If the active nematic is ``extensile,'' the rods slide past each other; however, if the active nematic is ``contractile,'' the activity drives the rods to slide towards each other.
These interactions result in extensile active nematics being unstable to bend distortions while contractile active nematics are unstable to splay distortions~\cite{RN171,RN170,RN11}.
Thus, the splay and bend instabilities make the homogeneously-aligned director state unstable such that the steady-state of an active nematic is often turbulent~\cite{RN7} and full of pairs of $s = \pm 1/2$ defects that are continuously created and annihilated~\cite{RN11,RN8,RN3,RN27,RN135,RN86}.
These turbulent dynamics are driven by the $s = +1/2$ defects, where the activity coupled with the polar structure of the $s = +1/2$ defects causes the  $s = +1/2$ defects to act like self-propelled particles, with the propulsion direction depending on the type of activity~\cite{RN11,RN8}.
In contrast, due to the three-fold symmetry of $s = -1/2$ defects, $s = -1/2$ defects are not driven by activity and are only advected by interactions between the defects as well as by interactions with the surface.\\

Recent experimental studies with active nematics yielding self-regulated behaviors such as self-sustained oscillations~\cite{RN9}, spontaneous formation of morphological features such as kinks and protrusions~\cite{RN9,RN3}, and undirected motility~\cite{RN9,RN3} have acted as as stimulus to investigate how biological functionality emerges from the interplay between activity, the geometry of the system, and the structure of the internal phase~\cite{RN160,RN51,RN10}.
In this context, defects in active nematics could be harnessed to achieve life-like functionality, as defects are extremely sensitive to the intrinsic geometry of the space they inhabit.
This is easily seen in the connection between the total topological charge on a surface and the topology of the surface, as expressed in the Poincr\'e-hopf Theorem seen in Eq.~[INTRO].
However, the theory of defects in curved spaces extends this connection beyond topology, predicting that the free-energy of a collection of defects is sensitive to the local geometry through the Gaussian curvature~\cite{RN42}.
Importantly, this implies that even though there have been new discoveries as a result of examining defect structures on spheres~\cite{RN45,RN106,RN26,RN110,RN105,RN76,RN101,RN165}, the Gaussian curvature of a sphere is constant such that the impact of curvature can only enter through the sphere radius.
This is is born out in the size-dependent onset of grain-boundary scars in colloidal crystals on the surface of emulsion drops~\cite{RN26,RN110} and the degeneracy in the orientation of the 4 $s = +1/2$ defects in nematic shells~\cite{RN45}.
Even when considering an active nematic on a sphere and the 4 $s = +1/2$ defects become mobile, the defect dynamics can be explained without any mention of Gaussian curvature~\cite{RN9}. \\

Despite theoretical interest in the interplay between varying Gaussian curvature and topological defects, with few exceptions~\cite{RN84,RN25,RN73,RN81}, there has been little experimental work with defects on surfaces where the Gaussian curvature is non-constant.
More notably, there has been no work where the defects have the option to explore regions with both positive and negative Gaussian curvature.
In this scenario with nematic order, the topological charge is expected to unbind such that $s = +1/2$ defects are attracted to regions of positive Gaussian curvature and $s=  -1/2$ defects are attracted to regions of negative Gaussian curvature~\cite{RN17,RN19,RN22}.
In this chapter we consider a 2D extensile active nematic composed of microtubules driven by kinesin motors fueled by adenosine triphosphate (ATP) on the surface of a torus.
Since the torus the torus has a handle, it has a genus of one and thus according to the Poincar\'e-Hopf Theorem written in Eq.~[INTRO] must have vanishing net topological charge.
We note that a nematic on a torus can satisfy this condition by either being defect-free or by having the same amount of positive and negative charge.
While strong spatial confinement or low activity can suppress the spontaneous formation of defect pairs in an active nematic~\cite{RN9,RN247}, we perform our experiments in the turbulent regime such that the torus is always populated with a sea of constantly moving $s = \pm 1/2$ defects that are dynamically created and annihilated.\\

Despite such chaotic and highly nonequilibrium dynamics, we find that on average, topological defects unbind and segregate in regions of oppositely-signed Gaussian curvature.
Notably, due to the chaotic dynamics and the averaging process, the topological charge ceases to be a discrete variable and instead approaches a continuous distribution.
In addition, contrary to equilibrium predictions~\cite{RN36,RN19,RN22,RN20,RN78}, we find that this active unbinding depends only on the local geometry and is independent of the system size and aspect ratio.
When also consider the defect species themselves and find that the average defect density depends inversely on Gaussian curvature over the majority of the toroidal surface, deviating only near the inside and outside of the handle.
Interestingly, near the inside and outside of the handle, we see that the defects themselves, on average, begin to exhibit orientational order.
While $s = +1/2$ defects have been shown to order themselves into a nematic phase in flat space, we find that the $s = +1/2$ defects on a torus exhibit polar order instead~\cite{RN27,RN6}.
We perform our studies with two different ATP concentrations and thus 2 different activities and find that the defect unbinding, the defect density, and the orientational order of the defects all depend on activity.
A numerical integration of the equation of motion of active nematic defects~\cite{RN11,RN8,RN9} performed by Luca Giomi and Dan Pearce at the University of Leiden confirms our experimental results, and further illustrating that the defect unbinding can even be suppressed in the limit of high activity.
Furthermore, by using topological defects as micro-rheological tracers and quantitatively comparing our experimental and theoretical results, we are able to estimate the Frank elastic constant, the active stress, and the defect mobility of a microtubule-kinesin active nematic liquid crystal.
Overall, out results not only confirm the theory of topological defects on curved surfaces, but also demonstrate the surprising phenomenology that arises from adding activity to the interplay between geometry, topology, and order. Out work thus provides insights into the physics of partially ordered active matter and introduces a new avenue for the quantitative mechanical characterization of active fluids.\\


\section{Making active nematic toroids}
\subsection{Active nematic formulation}
We use the mictotubule-kinesin active nematic system pioneered by the Dogic Group at Brandeis University as published in references~\cite{RN3,RN27,RN9,RN135,RN134}.
Microtubules are long, hollow rods that self-assemble from dimers of the $\alpha$- and $\beta$-tubulin proteins~\cite{RN248}.
The $\alpha$/$\beta$-tubulin dimers polymerize end-to-end to form long chains that then polymerize laterally to create cylindrical structures with a helical wrapping of $\alpha$- and $\beta$-tubulin chains~\cite{RN248,RN249}.
This structure gives microtubules a polarity with the (+) end associated with the exposed $\beta$-tubulin subunits and the (-) end associated with the exposed $\alpha$-tubulin subunits~\cite{RN248,RN249}.
While the microtubules serve as the mesogens of the active nematic, the activity comes from kinesin motor proteins bound in clusters to a streptavidin protein.
When in contact with a microtubule, kinesan ``walks'' along the microtubule in discrete steps from the (-) end towards the (+) end, hydrolyzing one ATP molecule into an adenosinediphosphate (ADP) molecule for every 8 nm step~\cite{RN250}.
Thus, a kinesin cluster in contact with two antiparallel microtubules will produce a sliding motion between the two microtubules as the motion of the kinesin motors on the microtubules will displace the two microtubules in opposite directions~\cite{RN4,RN3}.
Conversely, a kinesin cluster in contact with two parallel microtubules will simply walk along both microtubules in the same direction and thus produce no relative motion between the microtubules~\cite{RN4,RN3}. \\

Apart from the activity provided by the kinesin motors, the dominant interaction between the microtubules in an active nematic solution is the depletion interaction~\cite{RN251}, introduced via the presence of poly(ethylene glycol) (PEG, 20 kDa) as a depletant.
The depletion interaction ``bundles'' the microtubules together to form filaments that continuously grow due to the extensile interaction between microtubules provided by the kinesin motors~\cite{RN244,RN4,RN3}.
When the concentration of filaments in the active nematic solution is large enough, the filaments form a viscoelastic network~\cite{RN253,RN3}.
This network inhibits the growth of the filaments such that the filaments buckle at a critical length scale~\cite{RN253,RN3} and fracture into smaller fragments that recombine with other filaments, starting the growth-and-buckling process anew.
In the presence of a liquid-liquid interface, the depletion interaction drives the filaments to the interface between the active nematic solution and outer phase, increasing the concentration such that the filaments locally align and express the orientational order of the nematic phase~\cite{RN3,RN135,RN134}.
Thus, the filament direction serves as the director for the 2D active nematic localized to the interface.
We note that the growth-and-buckling phenomenology of the active filaments is a physical consequence of the extensile dynamics and in the nematic phase is directly responsible for the bend instability and associated defect formation in the  microtubule-kinesin active nematics.
As the filaments buckle and fracture, an $s = +1/2$ and $s = -1/2$ defect pair is produced, with the defects nucleating at opposite ends of the fracture line~\cite{RN3,RN11}.
Hence, with microtubules, kinesin-streptavidin complexes (K/SA), ATP, a depletant, and a liquid outer phase, we have the essential ingredients for an active nematic.\\

However, there are many more compounds in an active nematic solution that serve to allow for measurements~\cite{RN3,RN135}, prolong activity~\cite{RN3,RN135}, and allow for a silicone-based outer phases~\cite{RN135}.
The microtubules are fluorescently labeled such that the active nematic can be images with fluorescence or fluorescence confocal microscopy.
To prevent photobleaching and phototoxicity during imaging, the active solution contains trolox (Sigma, 238813) and two anti-oxidant solutions.
Anti-oxident solution 1 (AO1) is composed of glucose and dithiothreitol (DTT) and Anti-oxident solution 2 (AO2) is composed of glucose oxidase (Sigma, 238813) and catalase (Sigma, C40).
The active solution also includes an ATP regeneration system to keep the ATP concentration in an active nematic solution constant.
This system is driven by the enzyme mixture pyruvate kinase/lactate dehydrogenase (PK/LDH, Sigma, P-0294) which consumes phoshoenol pyruvate (PEP) to convert ADP to ATP at a rate faster than the K/SA hydrolyzes ATP to ADP~\cite{RN246}.
When driven the the liquid-liquid interface, the active filaments do not ever contact the outer phase.
Instead, the interface is packed with a suitable surfactant such that the active nematic filaments deplete to the polar portion of the surfactant molecule.
In fact, the surfactant is then moved along the interface by the motion of the active filaments such that the viscosity of the outer oil phase significantly affects the dynamics of an active nematic~\cite{RN135}.
For a silicone-based outer fluid, we include the triblock copolymer Pluronic F127 (F127) composed of 2 hydrophilic PEG block attached to a central hydrophobic poly(propylene oxide) (PPO) block like PEG-PPO-PEG~\cite{RN252}.
The solution is buffered using a specially designed microtubule buffer (M2B) to keep the enzymes and the motors in their preferred environments~\cite{RN3}.\\

For an experiment, we build the active nematic solution from stock solutions.
The stock solutions in their given compositions are {\bf bolded} such that PEP refers to the general compound while {\bf PEP} refers to the stock concentration/formulation as defined below:
\begin{itemize}
  \item[]{\bf M2B}: 80 mM 1,4-piperazinediethanesulphonic (PIPES) buffer~\cite{RN243} +2 mM MgCl$_2$ + 1mM egtazic acid (EGTA), pH 6.8
  \item[]{\bf PEP}: 200 mM in {\bf M2B}, pH 6.8.
  \item[]{\bf PK/LDH}: Used as purchased.
  \item[]{\bf ATP}: 50 mM in {\bf M2B}, pH 6.8
  \item[]{\bf DTT}: 0.5 mM in {\bf M2B}, pH 6.8
  \item[]{\bf trolox}: Used as purchased.
  \item[]{\bf MIX}: 67 mM MgCl$_2$ in {\bf M2B}
  \item[]{\bf PEG}: (20 kDa) 6\% w/w in {\bf M2B}
  \item[]{\bf F127}: 12\% w/w in {\bf M2B}
  \item[]{\bf glucose}: 300 mg/mL in 20 mM K$_2$HPO$_4$ + 70 mM KCl (pH 7.2)
  \item[]{\bf glucose oxidase}: 20 mg/mL in 20 mM K$_2$HPO$_4$ (pH 7.5)
  \item[]{\bf catalase}: 3.5 mg/mL in 20 mM K$_2$HPO$_4$ (pH 7.4)
  \item[]{\bf K/SA}: 0.175 mg/mL K401 + 0.1 mg/mL streptavidin (Invitrogen, S-888) + 12.5 mM imidazole (pH 6.8) + 1 mM MgCl$_2$ + 0.75 mM DTT + 12.5 mM ATP in {\bf M2B}. K401 consits of 401 amino acids of the N-terminal motor domain of \emph{D.~melanogaster} kinesin purified as previously published~\cite{RN3}.
  \item[]{\bf MT}: 8 mg/mL tubulin labeled with AlexaFluor 647 at 28\% labeling efficiency in {\bf M2B}. Tubulin was purified as previously published~\cite{RN243}.
\end{itemize}
The stock solutions were prepared by the Dogic Group at Brandeis and then shipped to Georgia Tech.
We pipetted the stock solutions as received into aliquots suitable to make 100 $\upmu$L of active nematic solution.
The aliquots are stored in a freezer at $-80^o$ C to prevent degradation of the active compounds.
Prior to each experiment, we remove a set of aliquots from the freezer, quickly thaw the aliquots to room temperature by holding them in our closed hands, and then place all the aliquots on ice except for the MT aliquot.
We leave the MT aliquot out at room temperature so that the tubulin can self-assemble into microtubules.
The polymerization of microtubules is temperature-dependent: with tubulin polymerizing to form microtubules at room temperature and microtubules depolymerizing into tubulin at low temperature~\cite{RN3}.
We then make the initial mixtures and the pre-solution as detailed in Table~\ref{t:3-recipe}, leaving the pre-solution on ice.
We wait at least 90 minutes after bringing the MT aliquot to room temperature before mixing the MT with the pre-solution to form the final active nematic solution as specified in Table~\ref{t:3-recipe}.
Note that we bring the pre-solution to room temperature before adding in the microtubules such that the microtubules do not depolymerize when they are added to the pre-solution.
Once we mix the final solution together, we perform our experiments and observe the active nematic until the activity ceases.

\begin{table}[ht]
  \centering
  \caption{Recipe for 100 $\upmu$L active nematic solution}
  \begin{tabular}{|r l|}
    \hline
    \multicolumn{2}{|c|}{Initial mixtures}\\
    \hline
    A01 & 1.5 $\upmu$L {\bf DTT} + 1.5 $\upmu$L {\bf glucose} \\
    A02 & 1.5 $\upmu$L {\bf glucose oxidase} + 1.5 $\upmu$L {\bf catalase} \\
    ATP2 & 2 $\upmu$L {\bf ATP} in:\\
    & \quad 18 $\upmu$L {\bf M2B} for a 10x dilution.\\
    & \quad 38 $\upmu$L {\bf M2B} for a 40x dilution.\\
    \hline
    \multicolumn{2}{|c|}{Pre-solution}\\
    \hline
    2.21 $\upmu$L & A01\\
    2.21 $\upmu$L & A02\\
    2.83 $\upmu$L & ATP2\\
    2.83 $\upmu$L & {\bf PK/LDH} \\
    4.83 $\upmu$L & {\bf MIX}\\
    10.00 $\upmu$L & {\bf trolox}\\
    13.33 $\upmu$L & {\bf PEP}\\
    13.33 $\upmu$L & {\bf PEG}\\
    16.67 $\upmu$L & {\bf F127}\\
    6.67 $\upmu$L & {\bf K/SA}\\
    8.40 $\upmu$L & {\bf M2B}\\
    \hline
    \multicolumn{2}{|c|}{Active nematic solution}\\
    \hline
    83.33 $\upmu$L & Pre-solution (full volume is 83.33 $\upmu$L)\\
    16.67 $\upmu$L & {\bf MT}\\
    \hline
  \end{tabular}
  \label{t:3-recipe}
\end{table}

\subsection{Flat sample confirmation}
\subsection{Making toroidal droplets}

\section{Imaging active nematic toroids}
\subsection{Confocal setup and parameters}
\subsection{Passing confocal data into MATLAB}

\section{Determining director and defects}
\subsection{Coherence-enhanced diffusion filtering}
\subsection{Calculating the director}
\subsection{Finding defect location and topological charge}
\subsection{Edge charge}

\section{Measuring surface curvature}
\subsection{The Weingarten matrix}
\subsection{Iteratively-reweighted least squares}
\subsection{Fitting the Weingarten matrix on a surface}
\subsection{Correcting the surface normal vectors}
\subsection{Finding the area element}
\subsection{Validation on test surfaces}
\subsection{Measuring the curvature of toroidal droplets}

\section{Defect charge and curvature}
\subsection{Finding regions of specific integrated Gaussian curvature}
\subsection{Measuring defect charge in a specified region}
\subsection{Time-averaged defect charge as a function of integrated Gaussian curvature: defect unbinding}

\section{Defect number and curvature}
\subsection{Average defect density}
\subsection{Defect number distributions}

\section{Defect orientation and curvature}

\section{Comparison with numerical calculations}
\subsection{Simulation details}
\subsection{Matching simulation parameters}
\subsection{Estimates of material parameters}

\section{Conclusions}
