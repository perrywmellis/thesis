%!TEX root = thesis.tex
\chapter{Introduction}\label{c:1}
At high temperature, an isolated fluid of spherical particles is a uniform system, i.e.\ the probability of finding a particle at a specific position is constant and only depends on the density of the particles.
As this system is invariant under all possible rotations and translations, the system has complete continuous translational and rotational symmetry.
If we lower the temperature, the particle collection will eventually develop order, where the individual particles have a preferred local arrangement.
For example, a crystalline phase has particles in periodic locations and thus is invariant only with respect to a discrete set of translations and rotations.
Thus the ``ordered'' phase has lower symmetry than the isotropic phase as the continuous symmetry of the isotropic phase was broken to achieve the ordered phase.
The advent of order as a result of continuous symmetry breaking is general and is a hallmark of transitions between a multitude of different phases.
For example, in three dimensions the isotropic-to-crystalline phase transition breaks continuous translational and rotational symmetry in all directions; the isotropic-to-smectic phase transition breaks continuous translation symmetry in one direction and continuous rotational symmetry in two directions; and the isotropic-to-nematic phase transition breaks no continuous translational symmetries, only breaking continuous rotational symmetry in at least two directions~\cite{RN175}.
With the development of order comes the rigidity needed to maintain that order.
This rigidity is not isotropic, but reflects the broken symmetries of the phase.
For example, crystalline and smectic materials do not flow easily due to their broken translational symmetries.
In contrast, nematics with their continuous translational symmetries flow far easier as the mesogens only need to maintain their orientation and not their position.
Similarly, a material with both a smectic phase and a crystalline phase will flow easier when it is in the smectic phase compared to when it is in the crystalline phase, reflecting the higher translational symmetry and thus smaller rigidity of the smectic phase.
This phenomenon is not limited to the rigidity; in general, the properties of a phase will reflect its symmetries.
For example, ordered materials often exhibit birefringence, where the index of refraction of the material depends on the polarization direction of the light incident on the material~\cite{RN175}.

Ordered materials can also have defects, defined generally as locations in the material where the preferred local arrangement is not satisfied.
Belying their name, defects are not necessarily undesireable; in fact, they can have important consequences for the physics of the phase.
For example, joining two crystalline domains of incompatible orientations results in defects forming a grain boundary between the two domains.
In a crystallline material, these grain boundaries affect the moduli of the material, and are even responsible for the phenomenon of ``work hardening,'' where plastically deforming and ``working'' the material strengthens the shear modulus.
The plastic deformations cause defects and grain boundaries to proliferate within the material; however, as the defect density rises, so does the energy required to generate a new defect, resulting in the increased resistance to deformation reflected in the strengthened shear modulus.
In addition, defects can also directly mediate phase transitions, as in the celebrated KTHNY theory of melting in 2D~\cite{RN161,RN163}.
Here, the increase in symmetry when a 2D crystalline phase melts to the isotropic phase is a two-step process.
First, pairs of disclinations~\cite{RN61,RN203}, or rotational defects, proliferate, driving the crystalline phase into a hexatic intermediate phase.
Second, these disclination pairs unbind, transforming the hexatic phase into the isotropic phase.
Furthermore, in soft matter, defects have been used to assemble hierarchical structures via preferential chemical linking as well as through long-range defect-defect interactions~\cite{RN43,RN50,RN150,RN157}.

Apart from their effect on physical systems, defects are fascinating objects in their own right as they are extremely sensitive to the intrinsic geometry of the space they inhabit.
Consider, for example, packing rods on a plane.
In order to maximize the volume fraction of the rods, the rods need to align along the same direction, breaking continuous rotational symmetry and yielding the the 2-fold order of the uniaxial nematic phase~\cite{RN204}.
We can characterize this preferred local alignment with a director, $\mathbf{n}$, where $\mathbf{n} = \mathbf{-n}$.
Clearly, it is easy to fill space on the plane with a homogeneous director field.
However, if we now try to pack rods on the surface of a sphere, for example along either the latitude or longitude lines on the Earth's globe, we see that there are defects in the order that correspond to the singular points at the poles, where $\mathbf{n}$ is undefined.
Since you ``cannot comb a hairy ball'', the presence of singularities is no accident --- it is a consequence of confining the nematic to the surface with the topology of a sphere~\cite{RN209,RN169}.

To formally relate the topology of the sphere with the presence of singularities, we first briefly introduce the necessary topological notions.
The sphere is an example of a differentiable surface that is compact, without boundary, and orientable.
Compact surfaces must both be bounded and contain their limit points; the boundary of a surface is defined as the set of points that can be approached from both within and outside the surface; and orientable surfaces have a defined normal vector everywhere.
For example, 2D Euclidean space, $\mathbb{R}^2$ is a surface but it is not bounded and thus is not compact.
With $\mathbf{r}$ a vector in $\mathbb{R}^2$, the 2D open unit disc $\mathbf{r} < d$ is not compact since it does not contain the circle with radius $d$.
However, the 2D unit disc $\mathbf{r} \leq d$ satisfies both conditions and thus is compact.
In addition, we see that $\mathbf{r} \leq d$ has a boundary $\partial \mathbf{r}$ at $|\mathbf{r}| = d$, as $\partial \mathbf{r}$ can be approached by points both within and outside the disc.
We call compact surfaces without a boundary closed surfaces.
Finally, all of these examples are also orientable; it is trivial to define the surface normal everywhere.

For a closed surface, we can write the Gauss-Bonnet theorem like,
\begin{equation}
  \chi = 2(1-g),\label{e:1-GB1}
\end{equation}
where $g$ is the genus, or number of handles, of the surface, and $\chi$ is the Euler characteristic of the surface.
The Euler characteristic is a topological invariant --- continuously deforming a surface, i.e. under a homeomorphism, does not change the Euler characteristic.
This implies that any surface with $g=0$ is topologically equivalent, or homoemorphic, to the sphere and thus also has $\chi=2$.
Thus, it is perhaps more appropriate to say, ``you cannot comb a hairy surface with $\chi=2$,'' such that packing rods on any handle-less closed surface will necessarily result in a director field that has singularities.
These singularities are disclinations; we characterize disclinations with their topological charge, measured by quantifying the director rotation along a closed path encircling the disclination~\cite{RN23,RN153,RN203}, defined as:
\begin{equation}
  s = \frac{1}{2 \pi}\oint_{\partial A} \textrm{d}\mathbf{r} \cdot \nabla\phi(\mathbf{r}),\label{eq:1-topCharge}
\end{equation}
with $\partial A$ giving the contour and the angle $\phi(\mathbf{r})$ parameterizing the director field, with $\mathbf{r}$ giving the position on the surface.
If $\phi(\partial A)$ represents the director orientations along the contour enclosing the disclination, we then see that the topological charge of the disclination is the winding number of $\phi(\partial A)$ on the unit circle, $\mathbb{S}_1$.
Hence, we see that regardless of aligning rods along the latitudes or the longitudes on the Earth's globe, the director along a contour encircling either pole rotates by $2 \pi$ such that both poles have charge $s = +1$, bringing the total charge on the surface to $+2$.
A formal statement connecting a vector field on a closed surface with the topology of the surface is given by the Poincar\'e-Hopf theorem:
\begin{equation}
  \sum\limits_i s_i = \chi,\label{e:1-PH}
\end{equation}
where $\chi$ is the Euler characteristic and $s_i$ is the topological charge of the $i^{\rm th}$ singularity, as defined in Eq.~\ref{eq:1-topCharge}.
Importantly, we now see that Eq.~\ref{eq:1-topCharge} is not limited to disclinations in director fields, but can be generally applied to disclinations in any system characterized by orientational order.
Note then, that the symmetry of the field on the surface dictates the minimum topological charge of any singularity as the field needs to continuous everywhere along $\partial A$.
For a nematic director field where $\mathbf{n} = -\mathbf{n}$, $s \in \mathbf{Z}/2$, while for a polar vector field, $s \in \mathbf{Z}$.
In addition, the Poincar\'e-Hopf Theorem only constrains the total charge on the surface, such that we could construct a director or a vector field on a topologically spherical surface with any combination of defects so long as the total topological charge was equal to $+2$.
Physical ordered systems constrained to a closed surface thus must minimize their free energy with respect to the constraint imposed by the topology.

Since the individual particles in an orientationally ordered phase prefer to align along each other, we can express the free-energy in the continuum limit as a functional of the orientation field,
\begin{equation}
  F_d = \frac{1}{2} k_F \int \textrm{d}^2\mathbf{r} \, |\nabla \phi(\mathbf{r})|^2,\label{e:2-XY}
\end{equation}
where $k_F$ is the elastic constant governing the cost of a distortion.
If we have polar order, then Eq.~\ref{e:2-XY} is the classical 2D X-Y model governing spins on a fixed lattice; for nematic order, Eq.~\ref{e:2-XY} is the 2D Frank-Oseen free energy~\cite{RN61} in the 1-constant approximation~\cite{RN33}.
For a nematic material on a sphere, it was predicted~\cite{RN42,RN104,RN43} that such a physical system minimizes its energy not with $2$ $s=+1$ defects but with $4$ $s=+1/2$ defects arranged on the vertices of a tetrahedron.
Prior work in our group addressed this situation experimentally using glass-based microfluidic devices to fabricate double emulsions~\cite{RN272}, with a shell of nematic liquid crystal (NLC) between an inner water droplet and an outer water continuous phase~\cite{RN105,RN45}.
By decreasing the osmotic pressure in the outer continuous phase of the nematic shells, swelling was induced in the inner water droplet, decreasing the thickness of the NLC shell to create a truly 2D nematic on the surface of a sphere~\cite{RN45}.
In this thin shell limit, our group found that the minimum energy state for a NLC on a sphere is indeed 4 $s = +1/2$ defects that arrange themselves on the vertices of a tetrahedron.

However, this configuration only exists in for thin enough shells, characterized by the relative shell thickness $h' = (R^{out}-R^{in})/R^{out}$, with $R^{out}$ the radius of the nematic shell and $R^{in}$ the radius of the inner water doplet.
As $h'$ increases, the nematic can no longer be considered 2D; in fact, there are now 2 spherical interfaces where the topological charge on the interface is constrained by the Poincar\'e-Hopf theorem with a bulk region filled with NLC between the the two surfaces.
Hence the situation is now 3D and the nematic directions are now parameterized by 2 angles.
If we naively assume that the director field is invariant along the radius of each emulsion droplet, each $s = +1/2$ disclination on the outer surface is now connected to a corresponding $s = +1/2$ disclination via a singular line that propagates through the bulk.
Thus, as $h'$ increases, so does the cost of each singular line.
Eventually, as $h'\lesssim 1/2$, the cost of propagating the singular lines through the bulk is too great and the 4 $s = +1/2$ disclinations on each surface transition to 2 $s = +1$ disclinations~\cite{RN105}, known colloquially as boojums~\cite{RN273}.
Unlike the $s = +1/2$ defects, the boojums are not connected by a singular line; instead, the director in the shell region acquires a component along the radius of the droplet, removing all the singularities in the bulk.
The increased energetic cost of a single boojum as compared to 2 $s = +1/2$ defect is compensated by the energetic decrease of removing the singular regions in the bulk of the shell.
This is an example of how a 3D system with a bulk volume minimizes its free-energy in the presence of 2D constraints.
Taking the limit that $h' \rightarrow 0$, the inner water droplet disappears and we have a single droplet of NLC in a continuous water phase, with the boojums on opposite poles and the director on the surface aligned along the longitude lines.
This director arrangement is the classic bipolar configuration.

Returning to order in 2D, it is clear from Eq.~\ref{e:2-XY} that a homogeneous orientation field corresponds to the zero energy configuration and any deviation from the homogeneous state costs energy.
As we have seen by packing rods onto a sphere, sometimes distortions are unavoidable due to the topological constraints imposed by the surface; however, ordered materials are also sensitive to the local geometry.
For example, consider once more rods packed on a plane with a homogeneous director field.
If we change the geometry and introduce a hemispherical ``bump'' into the plane, we see that we can no longer maintain a homogeneous director everywhere on the surface.
This inability to maintain the preferred local order due to the geometry of the surface is called ``geometrical frustration''.

We characterize the local geometry of a surface with its curvature; consider a planar curve $\mathbf{r}(s)$, with $s$ the arclength parameter constrained to lie in $\mathbb{R}^2$.
The unit tangent to the curve at $s$ is given by $\mathbf{T}(s) = \mathbf{r}'(s)$, and the local curvature by $\kappa(s) = |\textrm{d} \mathbf{T}(s)/\textrm{d}s^2| = | \textrm{d}^2\mathbf{r}/\textrm{d}s^2 |$.
Physically, $\kappa(s) = 1/R(s)$, where $R(s)$ is the radius of the circle that best approximates $\mathbf{r}(s)$ at $s$, also known as the ``radius of curvature'' at $s$.
For an orientable surface defined by $\mathbf{R}(u^1,u^2)$, with $(u^1,u^2) \in U$ local coordinates on the surface, we define a normal section at a point $\mathbf{r} \in U$ on the surface as the planar curve generated from intersecting $\mathbf{R}$ with a plane containing $\mathbf{k}(\mathbf{x})$, the unit normal to the surface at $\mathbf{x}$.
The curvature of a given normal section at $\mathbf{x}$ is thus called the normal curvature.
Taking all possible normal sections at $\mathbf{r}$, the principle curvatures $\kappa_1 (\mathbf{r})$ and $\kappa_2(\mathbf{r})$ are the maximum and minimum normal curvatures at $\mathbf{r}$, and their associated tangent vectors are called the principal normal directions.
We can now define the Gaussian curvature $K  = \kappa_1 \kappa_2)$ and the mean curvature $H = (1/2) (\kappa_1+\kappa_2)$.
Returning our example, we see that adding the hemisphere to the plane takes our previously flat surface with $K = H = 0$ everywhere and changes the geometry by adding curvature.
However, it is Gaussian curvature and not the mean curvature that is responsible for geometrical frustration.
This is evidenced by taking a flat plane and modulating it with a sine wave in one direction, keeping $K=0$ everywhere, but changing $H$.
We see that we can still maintain a homogeneous director field on the surface, in contrast to our example surface where $K \neq 0$ everywhere.
This reflects the fact that Gaussian curvature, also known as intrinsic curvature, is a property of the surface alone --- we can determine the Gaussian curvature of a surface without knowing anything about the space the surface is embedded in.
However, determining the mean, or extrinsic, curvature of a surface requires us to know what space the surface is embedded in.
Revisiting the Euler characteristic, we can now rewrite the Gauss-Bonnet theorem for a differentiable closed surface,
\begin{equation}
  \chi = \frac{1}{2 \pi} \int \, K \textrm{d}^2\mathbf{r} = 2(1-g),\label{e:1-GB2}.
\end{equation}
The Gaussian curvature provides the connection between the local geometry and the topology of a closed surface.

In addition, as demonstrated through our example of geometrical frustration, Gaussian curvature must also couple to the free-energy of an orientationally ordered phase confined to a surface.
In fact, for a topological charge density
\begin{equation}
  \rho(\mathbf{r}) = 2 \pi \sum\limits_i s_{(i)}\delta(\mathbf{r},\mathbf{r}_(i)),
\end{equation}
with disclinations indexed by $i$ possessing charge $s_(i)$ and position $\mathbf{r}_(i)$, and $\delta(\mathbf{r},\mathbf{r}_(i))$ the Kronecker delta, we can express Eq~\ref{e:2-XY} as~\cite{RN42,RN175,RN17}:
\begin{equation}
  F_d = -\frac{1}{2} k_f \int \textrm{d}^2\mathbf{r}\textrm{d}^2\mathbf{r}' G_L(\mathbf{r},\mathbf{r}') [\rho(\mathbf{r})-K(\mathbf{r})] [\rho(\mathbf{r}')-K(\mathbf{r}')],\label{e:1-TopTheoryofDefects}
\end{equation}
where $G_L$ is the Green's function of the Laplace-Beltrami operator, the standard Laplacian operator generalized to curved space.
It is worth mentioning that Eq.~\ref{e:1-TopTheoryofDefects} is identical to the Coulomb energy of a multicomponent plasma with charge density $\rho$ in a background of charge density $-K$~\cite{RN17}.
Thus, from the continuum theory, we see topological defects emerge as particle-like objects that possess topological charge and couple to the Gaussian curvature of the underlying substrate.
Importantly, this implies that even though there have been new discoveries as a result of examining defect structures on spheres~\cite{RN106,RN26,RN110,RN76,RN101,RN165}, including our prior work with spherical nematic shells highlighted earlier~\cite{RN45,RN105}, the Gaussian curvature and therefore the background topological charge density of a sphere is constant such that the impact of curvature can only enter through the sphere radius.
This is is born out in the size-dependent onset of grain-boundary scars in colloidal crystals on the surface of emulsion drops~\cite{RN26,RN110} and in the fact that the positions of the 4 $s = +1/2$ defects in nematic shells are due entirely to the defect-defect interactions alone with no influence from the Gaussian curvature.~\cite{RN45}.
If we want to investigate the role of curvature on ordered materials, we need to consider a space with varying Gaussian curvature and ideally, Gaussian curvature of different sign.

The simplest closed surface that satisfies these criteria is the torus, with $g = 1$ and thus $\chi = 0$.
In addition, we see that not only does a torus have sphere-like positive Gaussian curvature on the outer portion of the handle and saddle-like negative Gaussian curvature on the inner portion of the handle, but according to Eq~\ref{e:1-GB2}, the integrated Gaussian curvature over an entire torus must vanish.
Similarly, by Eq.~\ref{e:1-PH}, an orientationally ordered material on the surface of a torus must have vanishing topological charge and thus can support a defect-free configuration.
Prior work in our group investigated nematic order on the surface of a torus using toroidal droplets made from a NLC, with the NLC constrained to lie parallel to the interface between the toroidal droplet and the outer continuous medium~\cite{RN46,RN47}.
Like the thick spherical shells or a bipolar droplet, here we have a 3D nematic that must minimize its free-energy in the presence of a constraint on the topological charge on the 2D confining interface.
While we found that toroidal droplets have a defect-free ground state, the varying curvature still led to new insights about the role of a distortion called saddle-splay, revealing that saddle-splay drives the director in the plane of the surface to align along the smallest principle curvature~\cite{RN59}, yielding a doubly-twisted ground state in our toroidal droplets~\cite{RN46}.
We see that the amount of twist in the ground state can be controlled by the aspect ratio, or slenderness of the torus $\xi = R_0/a$, with $R_0$ the radius of the central circle of the torus and $a$ the tube radius of the torus.
In addition, this doubly-twisted state manifests itself through a spontaneous reflection symmetry breaking, where the free-energy resembles that of the Landau theory of magnetism.
In fact, in the cylindrical limit where $\xi \rightarrow \infty$, the free-energy of the doubly-twisted state has exactly the functional form of the Landau theory of magnetism.
This is an example of how confinement can affect a system, even in the absence of defects.

Up to this point we have considered how constraints imposed on a surface affect ordered materials constrained to lie on the surface or confined by the surface.
However, we can also use confinement to impose constraints on the bulk.
For example, consider the bipolar configuration where the director at the surface of the droplet is forced to lie tangential to the surface.
If we instead require the director at the surface of the droplet to be perpendicular to the interface, we see that there must be an irreducible singularity in the bulk.
These bulk singularities are known as hedgehogs and are characterized by their hedgehog charge,
\begin{equation}
  d = \frac{1}{4 \pi} \int_{\mathbb{S}^2} d\theta \: d\phi \: \mathbf{n} \cdot \left [ \partial_{\theta} \mathbf{n} \times \partial_{\phi} \mathbf{n} \right ],\label{e:1-HedgehogCharge}
\end{equation}
where the integral is taken over a spherical surface enclosing the defect, and $\theta$ and $\phi$ are, respectively, the polar and azimuthal angles on that surface~\cite{RN153}
Geometrically, the hedgehog charge relates the orientations of $\mathbf{n}$ taken on a surface that is topologically equivalent to a sphere enclosing the defect to the number of times the orientations of $\mathbf{n}$ cover the unit sphere~\cite{RN153}.
Thus, we see that confining a NLC to a volume that is topologically spherical with homeotropic boundary conditions must yield a total ``hedgehog charge'' $|d|=1$.

In this Thesis, we investigate the role of geometry on the interplay between order and confinement.
In Chapter 2, we begin with an introduction to the theory of nematic liquid crystals.
Then, in Chapter 3, we consider nematic order on the surface of a torus.
Due to the difficulty of creating a thin, stable, toroidal shell of a NLC, we use a polymeric nematic that self-assembles at the interface between two immiscible liquids.
Importantly, this lets us make stable toroidal droplets as in our previous work yet investigate 2D nematic order on the toroidal surface.
In addition, the nematic is active, such that the individual mesogens of the material have their own source of internal energy.
The activity then drives the nematic out of equilibrium at the individual particle level, generally filling the nematic with pairs of $s = \pm 1/2$ defects that are constantly in motion and constantly being created and annihilated.
Thus, even though we we have an inherently nonequilibrium materiel, we find that predictions built upon Eq.~\ref{e:1-TopTheoryofDefects} hold and that adding activity to order qualitatively resembles bringing an equilibrium system to the high temperature limit.

In Chapter 4, we consider NLC confined to toroidal droplets under homeotropic boundary conditions.
With the director constrained to lie perpendicular to the surface, the saddle-splay distortion does not affect the free-energy minimization.
However, we still find that a twisted ground-state configuration, where the amount of twist depends on $\xi$, eventually disappearing as $\xi \rightarrow \infty$.
Experiments with NLC confined under homeotropic boundary conditions to straight and bent cylindrical capillaries reveal that the twist is a response to the additional curvature induced when deforming a cylinder of homeotropic nematic into a torus.
In Chapter 5, we return to a spherical topology, confining NLC to a capillary bridge with homeotropic anchoring to study the influence of confinement shape on defect type.
Finally, in Chapter 6, we use toroids and spherical droplets to uncover the role of surface curvature of the anchoring of NLC.
In Chapter 7, we conclude.
